\newglossaryentry{default:NORTH-WESTERN-FRONTIER-PROVINCE-THE-NORTH-WEST-FRONTIER-PROVINCE-WAS-A-PROVINCE-OF-BRITISH-INDIA-IT-WAS-ESTABLISHED-IN-1901-AND-WAS-KNOWN-BY-THIS-NAME-UNTIL-2010-}{
	type=default,
	name={},
	description={North-Western Frontier Province The North-West Frontier Province was a province of British India. It was established in 1901 and was known by this name until 2010.}
}

\newglossaryentry{default:SEPARATE-ELECTORATE-THE-VOTING-POPULATION-OF-A-COUNTRY-OR-REGION-IS-DIVIDED-INTO-DIFFERENT-ELECTORATES-BASED-ON-CERTAIN-FACTORS-SUCH-AS-RELIGION-CASTE-GENDER-AND-OCCUPATION-HERE-MEMBERS-OF-EACH-ELECTORATE-VOTES-ONLY-TO-ELECTED-REPRESENTATIVES-FOR-THEIR-ELECTORATE-SEPARATE-ELECTORATES-ARE-USUALLY-DEMANDED-BY-MINORITIES-WHO-FEEL-IT-WOULD-OTHERWISE-BE-DIFFICULT-FOR-THEM-TO-GET-FAIR-REPRESENTATION-IN-GOVERNMENT-SEPARATE-ELECTORATE-FOR-MUSLIMS-MEANS-THAT-MUSLIMS-WILL-CHOOSE-THEIR-SEPARATE-LEADER-BY-SEPARATE-ELECTIONS-FOR-MUSLIMS-}{
	type=default,
	name={},
	description={Separate Electorate The voting population of a country or region is divided into different electorates, based on certain factors such as religion, caste, gender, and occupation. Here, members of each electorate votes only to elected representatives for their electorate. Separate electorates are usually demanded by minorities who feel it would otherwise be difficult for them to get fair representation in government. Separate electorate for Muslims means that Muslims will choose their separate leader by separate elections for Muslims.}
}
