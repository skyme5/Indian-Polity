\twocolumn

\chapter{Making of the Constitution}

\section{Demand For a Constituent Assembly}

It was in 1934-00-00 that the idea of a Constituent Assembly for India was put forward for the first time by M.N. Roy, a pioneer of communist movement in India. In 1935-00-00, the Indian National Congress (INC), for the first time, officially demanded a Constituent Assembly to frame the Constitution of free India. In 1938-00-00, Jawaharlal Nehru, on behalf of the INC declared that `the Constitution of free India must be framed, without outside interference, by a Constituent Assembly elected on the basis of adult franchise'.

The demand was finally accepted in principle by the British Government in what is known as the `August Offer' of 1940. In 1942-00-00, Sir Stafford Cripps, a member of the cabinet, came India with a draft proposal of the British Government on the framing of an independent Constitution to be adopted after the World War II. The Cripps Proposals were rejected by the Muslim League which wanted India to be divide into two autonomous states with two separate Constituent Assemblies. Finally, a Cabinet Mission\endnote was sent to India. While it rejected the idea of two Constituent Assemblies, it put forth a scheme for the Constituent Assembly which more or less satisfied the Muslim League.

\section{Composition of the Constituent Assembly}

The Constituent Assembly was constituted in 1946-11-00 under the scheme formulated by the Cabinet Mission Plan. The features of the scheme were

\paragraph{Features of the scheme}
\begin{enumerate}
  \item The total strength of the Constituent Assembly was to be 398. Of these, 296 seats were to be allotted to British India and 93 seats to the Princely States. Out of 296 seats allowed to the British India, 292 members were to be drawn from the eleven governors' provinces\endnote and four from the four chied commissioners' provinces\endnote, one from each.
  \item Each province and princely state (or group of states in case of small state) were to be allotted seats in proportion to their respective population. Roughly, one seat was to be allotted for every million population.
  \item Seats allocation to each British province were to be divide among the three principle communities - Muslims, Sikhs and general (all except Muslims and Sikhs), in proportion to their population.
  \item The representatives of each community were to be elected by members of that community in the provincial legislative assembly and voting was to be by the method of proportional representation by the means of single transferable vote.
  \item The representatives of princely states were to be nominated by the heads of the princely states.
\end{enumerate}

It is thus clear that the Constituent Assembly was to be a partly elected and partly nominated body. Moreover, the members were to be indirectly elected by the members of provincial assemblies, who themselves were elected on a limited franchise\endnote.

The election to the Constituent Assembly (for 296 seats allotted to the British Indian Provinces) were held in July-August 1946. The Indian National Congress won 208 seats, the Muslim Leagues 73 seats, and the small groups and independents got the remaining 15 seats. However, the 93 seats allotted to the princely states were not filled as they decided to stay away from the Constituent Assembly.

Although the Constituent Assembly was not directly elected by the people of India on the basis of adult franchise, the Assembly comprised representatives of all section of Indian Society - Hindus, Muslims, Sikhs, Parsis, Anglo-Indians, Indian Christians, SCs, STs including women of all these sections. The Assembly included all important personalities of India at that time, with the exception of Mahatma Gandhi.

\section{Working of the Constituent Assembly}

The Constituent Assembly held its first meeting on 1946-12-09. The Muslim League boycotted the meeting and insisted on a separate state of Pakistan. The meeting was thus attended by only 211 members. {\colord{}Dr Sachchidanand Sinha, the oldest member, was elected as the temporary President of the Assembly, following the French practice.}

Later, Dr. Rajendra Prasad was elected as the President of the Assembly. Similarly, both H.C. Mukherjee and V.T. Krishnamachi were elected as the Vice-Presidents of the Assembly. In other words, the Assembly had two Vice-Presidents.

\paragraph{Objectives Resolution}

On 1946-12-13, Jawaharlal Nehru moved the historic `Objectives Resolution' in the Assembly. It laid down the fundamentals and philosophy of the constitutional structure. It read:

\begin{enumerate}
  \item \textquotedblleft This Constituent Assembly declares its firm and solemn resolve to proclaim India as an Independent Sovereign Republic and to draw up for her future governance a Constitution:
  \item Wherein the territories that now comprise British India, the territories that now from the Indian States, and such other territories as are willing to be consituted into the independent sovereign India, shall be a United of them all; and
  \item wherein the said territories, whether with their present boundaries or with such others as may be determined by the Constitution, shall possess and retain the status of autonomous units together with residuary powers and exercise all powers and functions of Government and administration save and except such powers and functions as are vested in or assigned to the Union or as are inherent or implied in the Union or resulting therefrom; and
  \item wherein all power and authority of the Sovereign Independent India, its constituent parts and organs of Government are derived from the people; and
  \item wherein shall be guaranteed and secured to all the people of India justice, social, economic and political; equality of status and of opportunity, and before the law; freedom of thought, expression, belied, faith, worship, vocation, association and action, subject to law and public morality; and
  \item wherein adequate safeguards shall be provided for minorities, backwards and tribal areas, and depressed and other backward classes; and
  \item whereby shall be maintained the integrity of the territory of the Republic and its sovereign rights on land, sea and air according to justice and the law of civilized nations; and
  \item This ancient land attains its rightful and honored place in the world and makes its full and willing contribution to the promotions of world peace and the welfare of mankind.\textquotedblright
  
\end{enumerate}

This Resolution was unanimously adopted by the Assembly on 1947-01-22. It influenced the eventual shaping of the constitution through all its subsequent stages. Its modified version forms the Preamble of the present Constitution.

\section{Changes by the Independence Act}

The representatives of the princely states, who had stayed awat from the Constituent Assembly, gradually joined in. On 1947-04-28, representatives of the six states\endnote were part of the Assembly. After the acceptance of Mountbatten Plan of 1947-06-03 for a partition of the country, the representatives of most of the other princely states took their seats in the Assembly. The members of the Muslim League from the Indian Dominion also entered the Assembly.

The Indian Independence Act of 1947 made the following three changes in the position of the Assembly:

\begin{enumerate}
  \item The Assembly was made a fully sovereign body, which could frame any Constitution it pleased. The act empowered the Assembly to abrogate or alter any law made by the British Parliament in relation to India.
  \item The Assembly also became a legislative body. In other words, two separate functions were assigned to the Assembly, that is, making of a constitution for free India and enacting of ordinary laws for the country. These two tasks were to be performed on separate days. Thus, the Assembly became the first Parliament of free India (Dominion Legislature). Whenever the Assembly met as the Constituent body it was chaired by Dr. Rajendra Prasad and when it met as the legislative body\endnote, it was chaired by G.V. Mavalankar. These two functions continued till 1949-11-26, when the task of making the Constitution was over.
  \item The Muslim League members (hailing from the areas\endnote included in the Pakistan) withdrew from the Constituent Assembly for India. Consequently, the total strength of the Assembly came down to 299 as against 389 originally fixed in 1946 under the Cabinet Mission Plan. The strength of the Indian provinces (formerly British Provinces) was reduced from 296 to 299 and those of the princely states from 93 to 70. The state-wise membership of the Assembly as on 1947-12-31, is shown in \ref{tab:StateWiseMemberShipAssembly} at the end of this chapter.
\end{enumerate}

\section{Other Functions Performed}

In addition to the making of the Constitution and enacting of ordinary laws, the Constituent Assembly also performed the following functions:

\begin{enumerate}
  \item It ratified the India's membership of the Commonwealth in 1949-05-00.
  \item It adopted the national flag on 1950-01-24.
  \item It adopted the national anthem on 1950-01-24.
  \item It adopted the national song on 1950-01-24.
  \item It elected Dr. Rajendra Prasad as the first President of India on 1950-01-24.
\end{enumerate}

In all, the Constituent Assembly had 11 sessions over two years, 11 month and 18 days. The Constitution-makes had gone through the constitutions of about 60 countries, and the Draft Constitution was considered for 114 days. The total expenditure incurred on making the Constitution amounted to 64 lakh.

On 1950-01-24, the Constituent Assembly held its final session. It however, did not end, and continued as the provisional parliament of India from 1950-01-26 till the formation of new Parliament\endnote after the first general elections in 1951-52.

\section{Committees of the Constituent Assembly}

The Constituent Assembly appointed a number of committees to deal with different tasks of constitution-making. Out of these, eight were major committees and the others were minor committees. The names of the committees and their chairman are given below:

\subsection{Major Committees}

\begin{enumerate}
  \item Union Powers Committee - Jawaharlal Nehru
  \item Union Constitution Committee - Jawaharlal Nehru
  \item Provincial Constitution Committee - Sardar Vallabhbhai Patel
  \item Drafting Committee - Dr. B.R. Ambedkar
  \item Advisory Committee on Fundamental Rights, Minorities and Tribal and Excluded Areas - Sardar Vallabhbhai Patel. This committee had the following five sub-committees:
  \begin{enumerate}
    \item Fundamental Rights Sub-Committee - J.B. Kripalani
    \item Minorities Sub-Committee - H.C. Mukherjee
    \item North East Frontier Tribunal Areas Sub-Committee (with Assam Excluded \& Partially Excluded Areas) - Gopinath Bardoloi
    \item Excluded and Partially Excluded Areas Other than those in Assam Sub Committee - A.V. Thakkar
    \item North-West Frontier Tribal Areas Sub-Committee\endnote
  \end{enumerate}
  
  \item Rules of Procedure Committee - Dr. Rajendra Prasad
  \item States Committee (Committee for Negotiating with States) - Jawaharlal Nehru
  \item Steering Committee - Dr. Rajendra Prasad
\end{enumerate}

\subsection{Minor Committees}

\begin{enumerate}
  \item Finance and Staff Committee - Dr. Rajendra Prasad
  \item Credentials Committee - Alladi Krishnaswami Ayyar
  \item House Committee - B. Pattabhi Sitaramayya
  \item Order of Business Committee - Dr. K.M. Munshi
  \item Ad-hoc Committee on the National Flag - Dr. Rajendra Prasad
  \item Committee on the Functions of the Constituent Assembly - G.V. Mavalankar
  \item Ad-hoc Committee on the Supreme Court - S. Varadachari (Not an Assembly Member)
  \item Committee on Chief Commissioners’ Provinces - B. Pattabhi Sitaramayya
  \item Expert Committee on the Financial Provisions of the Union Constitution - Nalini Ranjan Sarkar (Not an Assembly Member)
  \item Linguistic Provinces Commission - S.K. Dar (Not an Assembly Member)
  \item Special Committee to Examine the Draft Constitution - Jawaharlal Nehru
  \item Press Gallery Committee - Usha Nath Sen
  \item Ad-hoc Committee on Citizenship - S. Varadachari
\end{enumerate}


\subsection{Drafting Committee}

Among all the committees of the Constituent Assembly, the most important committee was the Drafting Committee set up on 1947-08-29. It was this committee that was entrusted with the task of preparing a draft of the new Constitution. It consisted of seven members. They were

\begin{enumerate}
  \item Dr. B.R. Ambedkar ({\colord Chairmen})
  \item N. Gopalaswamy Ayyangar
  \item Alladi Krishnaswami Ayyar
  \item Dr. K.M. Munshi
  \item Sayed Mohammad Saadullah
  \item N. Madhava Rau (He replaced B L Mitter who resined due to ill-health)
  \item T T Krishnamachari (He replaced D P Khaitan who died in 1948)
\end{enumerate}

The Drafting Committee, after taking into consideration the proposal of the various committees, prepared the first draft of the Constitution of India, which was published in 1948-02-00. The people of India were given eight months to discuss the draft and propose amendments. In the light of the public comments, criticisms and suggestions, the Drafting Committee prepared a second draft, which was published in 1948-10-00.

The Drafting Committee took less than six months to prepare its draft. In all it say for 141 days.

\section{Enactment of the Constitution}

Dr. B.R. Ambedkar introduced the final draft of the Constitution in the Assembly on 1948-11-04 (first reading). The Assembly had a general discussion on it for five days (till 1948-11-09).

The second reading (clause by clause consideration) started on 1948-11-15 and ended on 1949-10-17. During this stage, as many as 7653 amendments were proposed and 2473 were actually discussed in the Assembly.

The third reading of the draft started on 1949-11-14. Dr. B.R. Ambedkar moved a motion - `the Constitution as settled by the Assembly be passed'. The motion on Draft Constitution was declared as passed on 1949-11-26, and received the signatures of the members and the president. Out of a total 299 members of the Assembly, only 284 were actually present on that day and signed the Constitution. This is also the date mentioned in the Preamble as the date on which the people of India in the Constituent Assembly adopted, enacted and gave to themselves this Constitution.

{\colord The Constitution as adopted on 1949-11-26, contained a Preamble, 395 Articles and 8 Schedules}. The Preamble was enacted after the entire Constitution was already enacted.

Dr. B.R. Ambedkar, the then Law Minister, piloted the Draft Constitution in the Assembly. He took a very prominent part in the deliberations of the Assembly. He was known for his logical, forceful and persuasive arguments on the floor of the Assembly. He is recognized as the `Father of the Constitution of India'. This brilliant writer, constitutional expert, undisputed leader of the scheduled castes and the `chief architech of the Constitution of India' is also known as a `Modern Manu'.

\section{Enforcement of the Constitution}

Some provisions of the Constitution pertaining to citizenship, elections, provisional parliament, temporary and transitional provisions, and short title contained in Articles 5, 6, 7, 8, 9, 60, 324, 366, 367, 379, 380, 388, 391, 392 and 393 came into force on 1949-11-26, itself.

The remaining provisions(the major part) of the Constitution came into force on 1950-01-26. This day is referred to in the Constitution as the `date of it commencement', and celebrated as the Republic Day.

January 26 was specifically chosen as the `date of commencement' of the Constitution because of its historical importance. It was on this day in 1930 that {\colord Purna Swaraj} day was celebrated, following the resolution of the Lahor Session (December 1929) of the INC.

With the commencement of the Constitution, the Indian Independence Act of 1947 and the Government of India Act of 1935, with all enactments amending or supplementing the latter Act, were repealed. The Abolition of Privy Council Jurisdiction Act (1949) was however continued.

\section{Criticism of the Constituent Assembly}

The critics have criticized the Constituent Assembly on various grounds. These are as follows

\begin{enumerate}
  \item \textbf{Not a Representative Body}: The critics have argued that the Constituent  Assembly was not a representative body as its members were not directly elected by the people of India on the basis of universal adult franchise.
  \item \textbf{Not a Sovereign Body}: The critics maintained that the Constituent Assembly was not a sovereign body as it was created by the proposals of the British Government. Further, they said that the Assembly held its sessions with the permission of the British Government.
  \item \textbf{Time Consuming}: According to the critics, the Constituent Assembly took unduly long time to make the Constitution. They stated that the framers of the American Constitution took only four months to complete their work. In this context, Naziruddin Ahmed, a member of the Constituent Assembly, coined a new name for the Drafting Committee to show his contempt for it. He called it a ``Drifting Committee''.
  \item \textbf{Dominated by Congress}: The critics charged that the Constituent Assembly was dominated by the Congress party. Granville Austin, a British Constitutional expert, remarked: `The Constituent Assembly was a one-party body in an essentially one-party country. The Assembly was the Congress and the Congress was India'\endnote.
  \item \textbf{Lawyer–Politician Domination}: It is also maintained by the critics that the Constituent Assembly was dominated by lawyers and politicians. They pointed out that other sections of the society were not sufficiently represented. This, to them, is the main reason for the bulkiness and complicated language of the Constitution.
  \item \textbf{Dominated by Hindus}: According to some critics, the Constituent Assembly was a Hindu dominated body. Lord Viscount Simon called it `a body of Hindus'. Similarly, Winston Churchill commented that the Constituent Assembly represented `only one major community in India'.
\end{enumerate}


\section{Important Facts}

\begin{enumerate}
  \item Elephant was adopted as the symbol (seal) of the Constituent Assembly.
  \item Sir B.N. Rau was appointed as the constitutional advisor (Legal advisor) to the Constituent Assembly.
  \item H.V.R. Iyengar was the Secretary to the Constituent Assembly.
  \item S.N. Mukerjee was the chief draftsman of the constitution in the Constituent Assembly.
  \item Prem Behari Narain Raizada was the calligrapher of the Indian Constitution. The original constitution was handwritten by him in a flowing italic style.
  \item The original version was beautified and decorated by artists from Shantiniketan including Nand Lal Bose and Beohar Rammanohar Sinha.
  \item Beohar Rammanohar Sinha illuminated, beautified and ornamented the original Preamble calligraph-ed by Prem Behari Narain Raizada.
  \item The calligraphy of the Hindi version of the original constitution was done by Vasant Krishan Vaidya and elegantly decorated and illuminated by Nand Lal Bose.
\end{enumerate}


\onecolumn

%January \d+, \d+|February \d+, \d+|March \d+, \d+|April \d+, \d+|May \d+, \d+|June \d+, \d+|July \d+, \d+|August \d+, \d+|September \d+, \d+|October \d+, \d+|November \d+, \d+|December \d+, \d+

\begin{longtable}[c]{@{}|p{1cm}|p{5.5cm}|p{5.5cm}|@{}}
  \caption{Allocation of seats in the Constituent Assembly of India (1946)}
  \label{tab:AllocationSeatsConstituentAssembly1946}\\
  \toprule
  Sl.No. & Areas & Seats \\* \midrule
  \endfirsthead
  %
  \multicolumn{3}{c}%
  {{\bfseries Table \thetable\ continued from previous page}} \\
  \toprule
  Sl.No. & Areas & Seats \\* \midrule
  \endhead
  %
  \bottomrule
  \endfoot
  %
  \endlastfoot
  %
  1. & British Indian Provinces (11) & 292 \\
  2. & Princely States (Indian States) & 93 \\
  3. & Chief Commissioners’ Provinces (4) & 4 \\
  \toprule
  & Total & 389 \\* \bottomrule
\end{longtable}


\begin{longtable}[c]{@{}|p{1cm}|p{5.5cm}|p{5.5cm}|@{}}
  \caption{Results of the Elections to the Constituent Assembly (July–August 1946)}
  \label{tab:ResultsElectionsConstituentAssembly1946}\\
  \toprule
  Sl.No. & Name of the Party & Seats won \\* \midrule
  \endfirsthead
  %
  \multicolumn{3}{c}%
  {{\bfseries Table \thetable\ continued from previous page}} \\
  \toprule
  Sl.No. & Name of the Party & Seats won \\* \midrule
  \endhead
  %
  1. & Congress & 208 \\
  2. & Muslim League & 73 \\
  3. & Unionist Party & 1 \\
  4. & Unionist Muslims & 1 \\
  5. & Unionist Scheduled Castes & 1 \\
  6. & Krishak – Praja Party & 1 \\
  7. & Scheduled Castes Federation & 1 \\
  8. & Sikhs (Non-Congress) & 1 \\
  9. & Communist Party & 1 \\
  10. & Independents & 8 \\
  \toprule
  & Total & 296\\* \bottomrule
\end{longtable}

\begin{longtable}[c]{@{}|p{1cm}|p{5.5cm}|p{5.5cm}|@{}}
  \caption{Community-wise Representation in the Constituent Assembly (1946)}
  \label{tab:CommunityRepresentationConstituentAssembly1946}\\
  \toprule
  Sl.No. & Community & Strength \\* \midrule
  \endfirsthead
  %
  \multicolumn{3}{c}%
  {{\bfseries Table \thetable\ continued from previous page}} \\
  \toprule
  Sl.No. & Community & Strength \\* \midrule
  \endhead
  %
  1. & Hindus & 163 \\
  2. & Muslims & 80 \\
  3. & Scheduled Castes & 31 \\
  4. & Indian Christians & 6 \\
  5. & Backward Tribes & 6 \\
  6. & Sikhs & 4 \\
  7. & Anglo-Indians & 3 \\
  8. & Parsees & 3 \\
  \toprule
  Total & 296 & 1\\* \bottomrule
\end{longtable}

\begin{longtable}[c]{@{}|p{1cm}p{1cm}|p{6cm}|p{4cm}|@{}}
  \caption{State wise Membership of the Constituent Assembly of India as on 1947-12-31}
  \label{tab:StateWiseMemberShipAssembly}\\
  \toprule
  \multicolumn{2}{|c|}{S.No.} & Name & No. of Members \\
  \bottomrule
  \endfirsthead
  %
  \multicolumn{4}{c}%
  {{\bfseries Table \thetable\ continued from previous page}} \\
  \toprule
  \multicolumn{2}{|c}{S.No.} & Name & No. of Members \\
  \bottomrule
  \endhead
  %
  \multicolumn{2}{|c}{
  \textbf{A.}} & \textbf{Provinces (Indian Provinces) - 299} &  \\\bottomrule
  & 1. & Madras & 49 \\
  & 2. & Bombay & 21 \\
  & 3. & West Bengal & 19 \\
  & 4. & United Provinces & 55 \\
  & 5. & East Punjab & 12 \\
  & 6. & Bihar & 36 \\
  & 7. & C.P. and Berar & 17 \\
  & 8. & Assam & 8 \\
  & 9. & Orissa & 9 \\
  & 10. & Delhi & 1 \\
  & 11. & Ajmer-Merwara & 1 \\
  & 12. & Coorg & 1 \\
  \toprule
  \multicolumn{2}{|c}{\textbf{B.}} & \textbf{Indian States (Princely States) - 70} &  \\\bottomrule
  & 1. & Alwar & 1 \\
  & 2. & Baroda & 3 \\
  & 3. & Bhopal & 1 \\
  & 4. & Bikaner & 1 \\
  & 5. & Cochin & 1 \\
  & 6. & Gwalior & 4 \\
  & 7. & Indore & 1 \\
  & 8. & Jaipur & 3 \\
  & 9. & Jodhpur & 2 \\
  & 10. & Kolhapur & 1 \\
  & 11. & Kotah & 1 \\
  & 12. & Mayurbhanj & 1 \\
  & 13. & Mysore & 7 \\
  & 14. & Patiala & 2 \\
  & 15. & Rewa & 2 \\
  & 16. & Travancore & 6 \\
  & 17. & Udaipur & 2 \\
  & 18. & Sikkim and Cooch Behar Group & 1 \\
  & 19. & Tripura, Manipur and Khasi States Group & 1 \\
  & 20. & U.P. States Group & 1 \\
  & 21. & Eastern Rajputana States Group & 3 \\
  & 22. & Central India States Group & 3 \\
  & 23. & Western India States Group & 4 \\
  & 24. & Gujarat States Group & 2 \\
  & 25. & Deccan and Madras States Group & 2 \\
  & 26. & Punjab States Group & 3 \\
  & 27. & Eastern States Group I & 4 \\
  & 28. & Eastern States Group II & 3 \\
  & 29. & Residuary States Group & 4 \\
  \toprule
  &  & Total & 299\\* \bottomrule
\end{longtable}

\begin{longtable}[c]{@{}|p{6cm}|p{6cm}|@{}}
  \caption{Sessions of the Constituent Assembly at a Glance}
  \label{tab:SessionsConstituentAssembly}\\
  \toprule
  Sessions & Period \\
  \bottomrule
  \endfirsthead
  %
  \multicolumn{2}{c}%
  {{\bfseries Table \thetable\ continued from previous page}} \\
	\toprule
  Sessions & Period \\
	\midrule
  \endhead
  %
  First Session & 9–23 December, 1946 \\
  Second Session & 20–25 January, 1947 \\
  Third Session & 28 April–2 May, 1947 \\
  Fourth Session & 14–31 July, 1947 \\
  Fifth Session & 14–30 August, 1947 \\
  Sixth Session & 27 January, 1948 \\
  Seventh Session & 4 November, 1948 to 8 January, 1949 \\
  Eighth Session & 16 May–16 June, 1949 \\
  Ninth Session & 30 July–18 September, 1949 \\
  Tenth Session & 6–17 October, 1949 \\
  Eleventh Session & 14–26 November, 1949\\* \bottomrule
\end{longtable}

\theendnotes
\cleardoublepage