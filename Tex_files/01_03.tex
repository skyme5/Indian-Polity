\twocolumn

\chapter{Salient Features of the Constitution}

\section{Introduction}

The Indian Constitution is unique in its contents and spirit. Through borrowed from almost every constitution of the world, the constitution of India has several salient features that distinguish it from the constitution of other countries.

It should be notes al the outset that a number of original features of the Constitution (as adopted in 1949) have undergone a substantial change, on account of several amendments, particularly 7th, 42nd, 44th, 73rd, 74th and 97th Amendments. In fact, the 42nd Amendment Act of 1976, is known as `Mini-Constitution' due to the important and large number of changes made by it in various parts of the Constitution. However, in the {\colord \textit{Keshavananda Bharti}} case\endnote (1973), the Supreme Court rules that the constitution power of Parliament under Article 368 does not enable it to alter the `basic structure' of the Constitution.

\section{Salient Features of the Constitution}

The salient features of the Constitution, as it stands today, are as follows

\subsection{Lengthiest Written Constitution}

Constitution are classified into written, like the American Constitution, or unwritten, like the British Constitution. The Constitution of India is the lengthiest of all the written constitutions of the world. It is very comprehensive, elaborate and detailed document.

Originally (1949), the Constitution contained a Preamble, 395 Articles (divided into 22 Parts) and 8 Schedules. Presently (2016), it consists of a Preamble, about 465 Articles (divided into 25 parts) and 12 Schedules\endnote. The various amendments carried out since 1951 have deleted about 20 Articles and one Part (VII) and added about 90 Articles, four Parts (IVA, IXA, IXB and XIVA) and four Schedules (9, 10, 11 and 12). No other Constitution in the world has so many Articles and Schedules\endnote.

Four factors have contributed to the elephantine size of our Constitution. They are

\renewcommand{\labelenumi}{\textbf{(\alph{enumi})}}
\begin{enumerate}
  \item Geographical factors, that is, the vastness of the country and its diversity.
  \item Historical factors, e.g., the influence of the Government of India Act of 1935, which was bulky.
  \item Single Constitution for both the Center and the states except Jammu and Kashmir\endnote.
  \item Dominance of legal luminaries in the Constituent Assembly.
\end{enumerate}

The Constitution contains not only the fundamental principles of governance but also detailed administrative provision. Further, those matters which in other modern democratic countries have been left to the ordinary legislation or established political conventions have also been included in the constitution document itself in India.

\subsection{Drawn from Various Sources}

The Constitution of India has borrowed most of its provisions from the constitutions of various other countries as well as from the Government of India Act\endnote of 1935. Dr. B.R. Ambedkar proudly acclaimed that the Constitution of India has been framed after `ransacking all the known Constitution of the World\endnote'.

The structural part of the Constitution is, to a large extent, derived from the Government of India Act of 1935. The philosophical part of the Constitution (the Fundamental Rights and the Directive Principles of State Policy) derive their inspiration from the American and Irish Constitution respectively. The political part of the Constitution (the principle of Cabinet Government and the relation between the executive and the legislature) have been largely drawn from the British Constitution\endnote.

The other provision of the Constitution have been drawn from the constitutions of Canada, Australia, Germany, USSR (now Russia), France, South Africa, Japan, and so on\endnote.

The most profound influence and material source of the Constitution is the Government of India Act, 1935. The Federal Scheme, Judiciary, Governors, emergency powers, the Public Service Commissions and most of the administrative details are drawn from this Act. More than half of the provisions of Constitution are identical to or bear a close resemblance to the Act of 1935\endnote.

\subsection{Blend of Rigidity and Flexibility}

Constitutions are also classified into rigid and flexible. A rigid Constitution is one that requires a special procedure for its amendment, as for example, the American Constitution. A flexible constitution, on the other hand, is one that can be amended in the same manner as the ordinary laws are made, as for example, the British Constitution.

The Constitution of India is neither rigid nor flexible but a synthesis of both. Article 368 provides for two types of amendments

\renewcommand{\labelenumi}{\textbf{(\alph{enumi})}}
\begin{enumerate}
  \item Some provisions can be amended by a special majority of the Parliament, i.e. a two-third majority of the members of each House present and voting, and a majority (that is, more than 50 percent), of the total membership of each House.
  \item Some other provisions can be amended by a special majority of the Parliament and with the ratification by half of the total states
\end{enumerate}

At the same time, some provisions of the Constitution can be amended by a simple majority of the Parliament in the manner of ordinary legislative process. Notable, these amendments do not come under Article 368.

\subsection{Federal System with Unitary Bias}

The Constitution of India establishes a federal system of government. It contains all the usual features of a federation, viz., two government, division of powers, written Constitution, supremacy of Constitution, rigidity of Constitution, independent judiciary and bicameralism.

However, the Indian Constitution also contains a large number of unitary or non-federal features, viz., a strong Center, single Constitution, single citizenship, flexibility of Constitution, integrated judiciary, appointment of state governor by the Centre, all-India services, emergency provisions, and so on.

Moreover, the term `Federation' has nowhere been used in the Constitution. Article 1, on the other hand, describes India as a `Union of States' which implies two things: one, Indian Federation is not the result of an agreement by the states; and two, no state has the right to secede from the federation.

Hence, the Indian Constitution has been variously described a `federal in form but unitary in spirit', `quasi-federl' by K.C. Wheare, `bargaining federalism' by Morris Jones, `co-operative federalism' by Granville Austin, `federation with a centralizing tendency' by Ivor Jennings, and so on.

\subsection{Parliamentary Form of Government}

The Constitution of India has opted for the British parliamentary System of Government rather than American Presidential System of Government. The parliamentary system is based on the principle of cooperation and co-ordination between the legislative and executive organs while the presidential system is based on the doctrine of separation powers between the two organs.

The parliamentary system is also known as the `Westminster'\endnote model of government, responsible government and cabinet government. The Constitution establishes the parliamentary system not only at the Centre but also in the states. The features of parliamentary government in India are

\renewcommand{\labelenumi}{\textbf{(\alph{enumi})}}
\begin{enumerate}
  \item Presence of nominal and real executives,
  \item Majority party rule,
  \item Collective responsibility of the executive to the legislature,
  \item Membership of the minister in the legislature,
  \item Leadership of the prime minister or the chief minister,
  \item Dissolution of the lower House (Lok Sabha or Assembly).
\end{enumerate}

Even though the Indian Parliamentary System is largely based on the British pattern, there are some fundamental difference between the two. For example, the Indian Parliament is not a sovereign body like the British Parliament. Further, the Indian State has an elected head (republic) while the British State has hereditary head (monarchy).

In a parliament system whether in India or Britain, the role of the Prime Minister has become so significant and crucial that the political scientists like to call its a `Prime Ministerial Government'.

\subsection{Synthesis of Parliamentary Sovereignty and Judicial Supremacy}

The doctrine of sovereignty of Parliament is associated with the British Parliament while the principle of judicial supremacy with that of the American Supreme Court.

Just as the Indian parliamentary system differs from the British system, the scope of judicial review power of the Supreme Court in India is narrower than that of what exists in US. This is because the American Constitution provides for `due process of law' against that of `procedure established by law' contained in the Indian Constitution (Article 21).

Therefor, the framers of the Indian Constitution have preferred a proper synthesis between the British principle of parliamentary sovereignty and the American principle of judicial supremacy. The Supreme Court, on the one hand, can declare the parliamentary laws as unconstitutional through its power of judicial review. The Parliament, on the other hand, can amend the major portion of the Constitution through its constituent power.

\subsection{Integrated and Independent Judiciary}

The Indian Constitution established a judicial system that is integrated as well a independent.

The Supreme Court stands at the top of the integrated judicial system in the country. Below it, there are high courts at the state level. Under a high court, there is a hierarchy of subordinate courts, that is, district courts and other lower courts. This single system of courts enforces both the central laws as well as the state laws, unlike in USA, where the federal laws are enforced by the federal judiciary and the state laws are enforced by the state judiciary.

The Supreme Court is a federal court, the highest court of appeal, the guarantor of the fundamental rights of the citizens and the guardian of the Constitution. Hence, the Constitution has made various provisions to ensure its independence — security of tenure of the judges, fixed service conditions for the judges, all the expenses of the Supreme Court charged on the Consolidated Fund of India, prohibition on the discussion on the conduct of judges in the legislatures, ban on practice after retirement, power to punish for its contempt vested in the Supreme Court, separation of the judiciary from the executive, and so on.

\subsection{Fundamental Rights}

Part III of the Indian Constitution guarantees six\endnote fundamental rights to all the citizens

\renewcommand{\labelenumi}{\textbf{(\alph{enumi})}}
\begin{enumerate}
  \item Right to Equality (Article 14-18),
  \item Right to Freedom (Article 19-22),
  \item Right against Exploitation (Article 23-24),
  \item Right to Freedom of Religion (Article 25-28),
  \item Cultural and Educational Right (Article 29-30), and
  \item Right to Constitutional Remedies (Article 32)
\end{enumerate}

The Fundamental Rights are meant for promoting the idea of political democracy. They operate as limitations on the tyranny of the executive and arbitrary laws of the legislature. They are justiciable in nature, that is they are enforceable by the courts for their violation. The aggrieved person can directly go to the Supreme Court which can issue the writs of {\colord habeas corpus}, {\colord mandamus}, prohibition, {\colord certiorari} and {\colord quo warranto} for the restoration of his rights.

However, the Fundamental Rights are not absolute the subject to reasonable restrictions. Further, they are not sacrosanct and can be curtailed or repealed by the Parliament through a constitutional amendment act. They can also be suspended during the operation of National Emergency except the rights guaranteed by Article 20 and 21.

\subsection{Directive Principles of State Policy}

According to Dr. B.R. Ambedkar, the Directive Principles of State Policy is a `novel feature' of the Indian Constitution. They are enumerated in Part IV of the Constitution. They can be classified into three broad categories — socialistic, Gandhian and liberal-intellectual.

The directive principles are meant for promoting the idea of social and economic democracy. They seek to establish a `welfare state' in India. However, unlike the Fundamental Rights, the directives are non-justifiable in nature, that is, they are not enforceable by the courts for their violation. Yet, the Constitution itself declares that `these principles are fundamental in the governance of the country and it shall be the duty of the state to apply these principles in making laws'. Hence, they impose a moral obligation on the state authorities for their application. But, the real force (sanction) behind them is political, that is, public opinion.

In the {\colord \textit{Minerva Mills}} case\endnote (1980), the Supreme Court held that `the Indian Constitution is founded on the bedrock of the balance between the Fundamental Rights and the Directive Principles'.

\subsection{Fundamental Duties}

The original constitution did not provide for the fundamental duties of the citizens. These were added during the operation of internal emergency(1975-77) by the 42nd Constitutional Amendment Act of 1976, on the recommendation of the Swaran Singh Committee. The 86the Constitutional Amendment Act of 2002, added one more fundamental duty.

The Part IV-A of the Constitution (which consists of only one Article-51 A) specifies the eleven Fundamental Duties,

\renewcommand{\labelenumi}{\textbf{(\alph{enumi})}}
\begin{enumerate}
  \item to respect the Constitution, national flag and national anthem;
  \item to protect the sovereignty, unity and integrity of the country;
  \item to promote the spirit of common brotherhood amongst all the people,
  \item to preserve the rich heritage of our composite culture and so on.
\end{enumerate}

The fundamental duties serve as reminder to citizens that while enjoying their rights, they have also to be quite conscious of duties they owe to their country, their society and to their fellow-citizens. However, like the Directive Principles, the duties are also non-justiciable in nature.


\subsection{A Secular State}

The Constitution of India stands for a secular state. Hence, it does not uphold any particular religion as the official religion of the Indian State. The following provisions of the Constitution reveal the secular character of the Indian State

\renewcommand{\labelenumi}{\textbf{(\alph{enumi})}}
\begin{enumerate}
  \item The term `secular' was added to the Preamble of the Indian Constitution by the 42nd Constitutional Amendment Act of 1976.
  \item The Preamble secures to all citizens of Indian liberty of belief, faith and worship.
  \item The State shall not deny to any person equality before the law or equal protection of the laws (Article 14).
  \item The State shall not discriminate against any citizen on the ground of religion (Article 15).
  \item Equality of opportunity for all citizens in matters of public employment (Article 16).
  \item All persons are equally entitled to freedom of conscience and the right to freely profess, practice and propagate any religion (Article 26).
  \item Every religious denomination or any of its section shall have the right to manage its religious affairs (Article 26).
  \item No person shall be compelled to pay any taxes for the promotion of particular religion (Article 27).
  \item No religious instruction shall be provided in any educational institution maintained by the state (Article 28).
  \item Any section of the citizens shall have the right to conserve its distinct language, script or culture (Article 29).
  \item All minorities shall have the right to establish and administer educational institutions of their choice (Article 30).
  \item The State shall endeavor to secure for all the citizens a Uniform Civil Code (Article 44).
\end{enumerate}

The Western concept of secularism connotes a complete separation between the religion (the church) and the state (the politics). This negative concept of secularism is inapplicable in the Indian situation where the society if multi-religious. Hence, the Indian Constitution embodies the positive concept of secularism, i.e., giving equal respect to all religions or protecting all religions equally.

Moreover, the Constitution has also abolished the old system of communal representation\endnote, that is, reservation of seats in the legislature on the basis of religion. However, it provides for the temporary reservation of seats for the scheduled castes and scheduled tribes to ensure adequate representation to them.

\subsection{Universal Adult Franchise}

The Indian Constitution adopts universal adult franchise as a basis of elections to the Lok Shabha and the state legislative assemblies. Every citizen who is not less then 18 years of age has a right to vote without any discrimination of caste, race, religion, sex, literacy, wealth, and so on. The voting age was reduced to 18 years from 21 years in 1989 by the 61st Constitutional Amendment Act of 1988.

The Introduction of universal adult franchise by the Constitution-makers was bold experiment and highly remarkable in view of the vast size of the country, its huge population, high poverty, social inequality and overwhelming illiteracy.\endnote

Universal adult franchise makes democracy broad-based, enhances the self-respect and prestige of the common people, upholds the principle of equality, enables minorities to protect their interests and opens up new hopes and vistas for weaker sections.

\subsection{Single Citizenship}

Through the Indian Constitution is federal and envisages a dual polity (Centre and states), it provides for only a single citizenship, that is, the Indian citizenship.

The countries like USA, on the other hand, each person is not only a citizen of USA but also a citizen of the particular state to which he belongs. Thus, the owes allegiance to both and enjoys dual sets of rights — one conferred by the National government and another by the state government.

In India, all citizens irrespective of the state in which they are born or reside enjoy the same political and civil rights of citizenship all over the country and no discrimination is made between them exception in few cases like tribal areas, Jammu and Kashmir, and so on.

Despite the constitutional provision for a single citizenship and uniform rights for all the people, India has been witnessing the communal riots, class conflicts, caste wars, linguistic clashes and ethnic disputes. This means that the cherished goal of the Constitution-makers to build and united and integrated Indian nation has not been fully realized.

\subsection{Independent Bodies}

The Indian Constitution not only provides for the legislative, executive and judicial organs of the government (Central and state) but also established certain independent bodies. They are envisaged by the Constitution as the bulwarks of the democratic system of Government in India. These are

\renewcommand{\labelenumi}{\textbf{(\alph{enumi})}}
\begin{enumerate}
  \item Election Commission to ensure free and fair elections to the Parliament, the state legislatures, the office of President of India and the office of Vice-president of India.
  \item Comptroller and Auditor-General of India to adult the accounts of the Central and state governments. He acts as the guardian of public purse and comments on the legality and propriety of government expenditure.
  \item Union Public Service Commission to conduct examinations for recruitment to all-India services\endnote and higher Central services and to advise the President on disciplinary matters.
  \item State Public Service Commission in every state to conduct examinations for recruitment to state services and to advice the governor on disciplinary matters.
\end{enumerate}

The Constitution ensures the independence of these bodies through various provision like security of tenure, fixed service conditions, expenses being charges on the Consolidated Funds of India, and so on.

\subsection{Emergency Provisions}

The Indian Constitution contains elaborate emergency provisions to enable the President to meet any extraordinary situation effectively. The rationality behind the incorporation of these provision is to safeguard the sovereignty, unity, integrity and security of the country, the democratic political system and the Constitution.

The Constitution envisages three types of emergencies, namely

\renewcommand{\labelenumi}{\textbf{(\alph{enumi})}}
\begin{enumerate}
  \item National emergency on the ground of war or externam aggression or armed rebellion\endnote (Article 352)
  \item State emergency (President's Rule) on the ground of failure of Constitutional machinery in the states (Article 356) or failure to comply with directions of the Centre (Article 365); and
  \item Financial emergency on the ground of threat to the financial stability or credit of India (Article 360).
\end{enumerate}

During an emergency, the Central Government becomes all-powerful and the states go into the total control of the centre. It converts the federal structure into a unitary one without a formal amendment of the Constitution. This kind of transformation of the political system from federal (during normal times) to unitary (during emergency) is unique feature of the Indian Constitution.

\subsection{Three-tier Government}

Originally, the Indian Constitution, like any other federal constitution, provided for a dual polity and contained provision with regard to the organization and powers of the Centre and the states. Later, the 73rd and 74the Constitutional Amendment Act of 1992, has added a third-tier of government (i.e., local) which is not found in any other Constitution of the world.

The 73rd Amendments Act of 1992 gave constitutional recognition to the Panchayat (rural local governments) by adding a new Part IX\endnote and a new Schedules 11 to the Constitution. Similarly, the 74the Amendment Act of 1992, gave constitutional recognition to the municipalities (urban local government) by adding a new Part IX-A\endnote and a new Schedule 12 to the Constitution.

\subsection{Co-operative Societies}

The $97^{\text{th}}$ Constitutional Amendment Act of 2011, gave a constitutional status and protection to co-operative societies. In this context, it made the following three changes in the Constitution

\begin{enumerate}
  \item It made the right to form co-operative societies a fundamental right (Article 19).
  \item It include a new Directive Principle of State Policy on promotion of co-operative societies (Article 43-B).
  \item It added a new Part IX-B in the Constitution which is entitled as ``The Co-operative societies'' (Article 243-ZH to 243-ZT).
\end{enumerate}

The new Part IX-B contains various provision to ensure that the co-operative societies in the country function in a democratic, professional, autonomous and economically sound manner. It empowers the Parliament in respect of multi-state cooperative societies and the state legislature in respect of other co-operative societies to make the appropriate law.

\section{Criticism of the Constitution}

The Constitution of India as framed and adopted by the Constituent Assembly of India has been criticized on the following grounds

\paragraph{A Borrowed Constitution}

The critics opined that the Indian Constitution contains nothing new and original. They described it as a `borrowed constitution' or a `bag of borrowings' or a `hotch-potch constitution' or a `patchwork' of several documents of the world constitutions. However, this criticism is unfair and illogical. This is because, the framers of the constitution made necessary modification in the features borrowed from other constitutions for their suitability to the Indian conditions, at the same time avoiding their faults.

While answering the above criticism in the Constituent Assembly, Dr. B.R. Ambedkar, the Chairman of the Drafting Committee, said: ``One likes to ask whether there can be antthin new in Constitution framed at this hour in the history of the world. More than hundred years have rolled over when the first written constitution was drafted. It has been followed by many countries reducing their constitution to writing. What the scope of a constitution should be has long been settled. Similarly, what are the fundamentals of a constitution are recognized all over the world. Given these facts, all constitutions in thir main provision must look similar. The only new things, if there can be any, in a constitution framed so late in the day are the variations made to remove the faults and to accommodate it to the needs of country. The charge of providing a blind copy of the constitutions of other countries is based, I am sure, on an inadequate study of the Constitution''\endnote.

\paragraph{A Carbon Copy of the 1935 Act}

The critics said that the framers of the constitution have included a large number of the provision of the Government of India Act of 1935 into the Constitution of India. Hence, they called the constitutions a ``Carbon Copy of the 1935 Act'' or an ``amended version of the 1935 Act''. For example, N. Shrinivasan observed that the Indian Constitution is ``both in language and substance a close copy of the Act of 1935''. Similarly, Sir Ivor Jennings, a British Constitutionalist, said that ``the constitution derives directly from the Government of India Act of 1935 from which, in fact, many of its provisions are copied almost textually''.

Further, P.R. Deshmukh, a member of the Constituent Assembly, commented that ``the constitution is essentially the Government of India Act of 1935 with only adult franchise added''.

The same Dr. B.R. Ambedkar answered the above criticism in the Constituent Assembly in the following way: ``As to the accusation that the Draft Constitution has reproduced a good part of the provision of the Government of India Act, 1935, I make no apologies. There is nothing to be ashamed of in borrowing. It involves no plagiarism. Nobody hold any patent rights in the fundamental ideas of a Constitution. What I am sorry about is that the provision taken from the Government of India Act, 1935, relate mostly to the details of administration''\endnote.

\paragraph{Un-Indian or Anti-India}

According to the critics, the Indian Constitution is `un-Indian' or `anti-Indian' because it does not reflect the political tradition and the spirit of India. They said that the foreign nature of the Constitution makes it unsuitable to the Indian situation or unworkable in India. In this context, K. Hanumanthaiya, a member of the Constituent Assembly, commented: ``We wanted the music of Veena or Sitar, but here we have the music of an English band. That was because out constitution-makers were educated that way''\endnote. Similarly, K. Hanumanthaiya, another member of the Constituent Assembly, criticized the constitution as ``slavish imitation of the west, much more - a slavish surrender to the west''\endnote. Further, Laxminarayan Sahu, also a member of Constituent Assembly, observed: ``The ideals on which this draft constitution is framed have no manifest relation to the fundamental spirit of India. This constitution would not prove suitable and would break down soon after being brought into operation''\endnote.

\paragraph{An Un-Gandhian Constitution}

According to the critics, the Indian Constitution is Un-Gandhian because it does not contain the philosophy and ideals of Mahatma Gandhi, the father of the Indian Nation. They opined that the Constitution should have been raised and build upon village panchayats. In this context, the same member of the Constituent Assembly, K. Hanumanthaiya, said: ``That is exactly the kind of Constitution Mahatma Gandhi did not want and did not envisage''.\endnote T. Prakasam, another member of the Constituent Assembly, attributed this lapse to Dr. B.R. Ambedkar's non-participation in the Gandhian movement and the antagonism towards Gandhian ideas\endnote.

\paragraph{Elephantine Size}

The critics stated that the Indian Constitution is too bulky and too detailed and contains some unnecessary elements. Sir Ivor Jennings, a British Constitutionalist, observed that the provisions borrowed were not always well-selected and that the constitution, generally speaking, was too long and complicated.\endnote

In this context, H.V. Kamath, a member of the Constituent Assembly, commented: ``The emblem and the crest we have selected for our assembly is an elephant. It is perhaps in consonance with that our constitution too is the bulkiest in the world has produced''.\endnote He also said: ``I am sure, the House does not agree that we should make the Constitution an elephantine one''.\endnote

\paragraph{Paradise of the Lawyers}

According to the critics, the Indian Constitution is too legalistic and very complicated. They opined that the legal language and phraseology adopted in the constitution makes it a complex document. The same Sir Ivor Jennings called it a ``lawyer's paradise''.

In this context, H.K. Maheswari, a member of the Constituent Assembly observed: ``The draft tends to make people more litigious, more inclined to go to courts, less truthful and less likely to follow the methods of truth and non-violence. If I may so, the Draft is really a lawyer's paradise. It opens up vast avenues of litigation and will give our able and ingenious lawyers plenty of work to do''.\endnote

Similarly, P.R. Deshmukh, another member of the Constituent Assembly, said: ``I should, however, like to say that the draft of the articles that have been brought before the House by Dr. B.R. Ambedkar seems to my mind to be far too ponderous like the ponderous tomes of a law manual. A document dealing with a constitution hardly uses so much of padding and so much of verbiage. Perhaps it is difficult for them to compose a document which should be, to my mind, not a law but a socio-political document, a vibration, pulsating and life giving document. But to our misfortune, that was not to be, and we have been burdened with so much of words, words and words which could have been very easily eliminated''.\endnote

\clearpage

\onecolumn

\begin{longtable}[c]{@{}|p{1cm}|p{7cm}|p{4cm}|@{}}
  \caption{The Constitution of India at a Glance}
  \label{tab:TheConstitutionofIndiaataGlance}\\
  \toprule 
	Parts & Subject Matter & Articles Covered \\
  \bottomrule
  \endfirsthead
  %
  \multicolumn{3}{c}%
  {{\bfseries Table \thetable\ continued from previous page}} \\
  \toprule 
	Parts & Subject Matter & Articles  Covered \\ 
	\bottomrule
  \endhead
  %
  I & The Union and its territory & 1 to 4 \\\midrule
  II & Citizenship & 5 to 11 \\\midrule
  III & Fundamental Rights & 12 to 35 \\\midrule
  IV & Directive Principles of State Policy & 36 to 51 \\\midrule
  IV-A & Fundamental Duties & 51-A \\\midrule
  V & The Union Government & 52 to 151 \\
  & Chapter I – The Executive & 52 to 78 \\
  & Chapter II – Parliament & 79 to 122 \\
  & Chapter III – Legislative Powers of President & 123 \\
  & Chapter IV – The Union Judiciary & 124 to 147 \\
  & Chapter V – Comptroller and Auditor-General of Ind & 148 to 151 \\\midrule
  VI & The State Governments & 152 to 237 \\
  & Chapter I – General & 152 \\
  & Chapter II – The Executive & 153 to 167 \\
  & Chapter III – The State Legislature & 168 to 212 \\
  & Chapter IV – Legislative Powers of Governor & 213 \\
  & Chapter V – The High Courts & 214 to 232 \\
  & Chapter VI – Subordinate Courts & 233 to 237 \\\midrule
  VII & The States in Part B of the First Schedule (deleted) & 238 (deleted) \\\midrule
  VIII & The Union Territories & 239 to 242 \\\midrule
  IX & The Panchayat & 243 to 243-0 \\\midrule
  IX-A & The Municipalities & 243-P to 243-ZG \\\midrule
  IX-B & The Co-operative Societies & 243-ZH to 243-ZT \\\midrule
  X & The Scheduled and Tribal Are & 244 to 244-A \\\midrule
  XI & Relations between the Union and the States & 245 to 263 \\
  & Chapter I – Legislative Relations & 245 to 255 \\
  & Chapter II – Administrative Relations & 256 to 263 \\\midrule
  XII & Finance, Property, Contracts and Suits & 264 to 300-A \\*\midrule
  & Chapter I – Finance & 264 to 291 \\
  & Chapter II – Borrowing & 292 to 293 \\
  & Chapter III – Property, Contracts, Rights, &  \\
  & Liabilities, Obligations and Suits & 294 to 300 \\
  & Chapter IV – Right to Proper & 300-A \\ \midrule
  XIII & Trade, Commerce and Intercourse within the Territory of India & 301 to 307 \\\midrule
  XIV & Services under the Union and the Stat & 308 to 323 \\
  & Chapter I – Services & 308 to 314 \\
  & Chapter II – Public Service Commissions & 315 to 323 \\\midrule
  XIV-A & Tribunals & 323-A to 323-B \\\midrule
  XV & Elections & 324 to 329-A \\\midrule
  XVI & Special Provisions relating to Certain Class & 330 to 342 \\\midrule
  XVII & Official Languages & 343 to 351 \\
  & Chapter I – Language of the Union & 343 to 344 \\
  & Chapter II – Regional Languages & 345 to 347 \\
  & Chapter III — Language of the Supreme Court, High Courts, and so & 348 to 349 \\
  & Chapter IV — Special Directives & 350 to 351 \\\midrule
  XVIII & Emergency Provisions & 352 to 360 \\\midrule
  XIX & Miscellaneous & 361 to 367 \\\midrule
  XX & Amendment of the Constitution & 368 \\\midrule
  XXI & Temporary, Transitional and Special Provisions & 369 to 392 \\\midrule
  XXII & Short title, Commencement, Authoritative Text in Hindi and Repeals & 393 to 395\\\bottomrule
\end{longtable}

\clearpage
\begin{longtable}[c]{@{}|p{2cm}|p{10cm}|@{}}
  \caption{Important Articles of the Constitution at a Glance}
  \label{tab:articlesnumberinfo}\\
  \toprule
  Articles & Deals with \\* \midrule
  \endfirsthead
  %
  \multicolumn{2}{c}%
  {{\bfseries Table \thetable\ continued from previous page}} \\
  \toprule
  Articles & Deals with \\* \midrule
  \endhead
  %
  1 & Name and territory of the Union \\ \midrule
  3 & Formation of new states and alteration of areas, boundaries or names of existing states \\ \midrule
  13 & Laws inconsistent with or in derogation of the fundamental rights \\ \midrule
  14 & Equality before law \\ \midrule
  16 & Equality of opportunity in matters of public employment \\ \midrule
  17 & Abolition of untouchability \\ \midrule
  19 & Protection of certain rights regarding freedom of speech, etc. \\ \midrule
  21 & Protection of life and personal liberty \\ \midrule
  21 & A Right to elementary education \\ \midrule
  25 & Freedom of conscience and free profession, practice and propagation of religion \\ \midrule
  30 & Right of minorities to establish and administer educational institutions \\ \midrule
  31—C & Saving of laws giving effect to certain directive principles \\ \midrule
  32 & Remedies for enforcement of fundamental rights including writs \\ \midrule
  38 & State to secure a social order for the promotion of welfare of the people \\ \midrule
  40 & Organisation of village panchayats \\ \midrule
  44 & Uniform civil code for the citizens \\ \midrule
  45 & Provision for early childhood care and education to children below the age of 6 years. \\ \midrule
  46 & Promotion of educational and economic interests of scheduled castes, scheduled tribes and other weaker sections \\ \midrule
  50 & Separation of judiciary from executive \\ \midrule
  51 & Promotion of international peace and security \\ \midrule
  51 & A Fundamental duties \\ \midrule
  72 & Power of president to grant pardons, etc., and to suspend, remit or commute sentences in certain cases \\ \midrule
  74 & Council of ministers to aid and advise the president \\
  78 & Duties of prime minister as respects the furnishing of information to the president, etc. \\ \midrule
  110 & Definition of Money Bills \\ \midrule
  112 & Annual financial statement (Budget) \\ \midrule
  123 & Power of president to promulgate ordinances during recess of Parliament \\ \midrule
  143 & Power of president to consult Supreme Court \\ \midrule
  155 & Appointment of governor \\ \midrule
  161 & Power of governor to grant pardons, etc., and to suspend, remit or commute sentences in certain cases \\ \midrule
  163 & Council of ministers to aid and advise the governor \\ \midrule
  167 & Duties of chief minister with regard to the furnishing of information to governor, etc. \\ \midrule
  169 & Abolition or creation of legislative councils in states \\ \midrule
  200 & Assent to bills by governor (including reservation for President) \\ \midrule
  213 & Power of governor to promulgate ordinances during recess of the state legislature \\ \midrule
  226 & Power of high courts to issue certain writs \\ \midrule
  239 & AA Special provisions with respect to Delhi \\ \midrule
  249 & Power of Parliament to legislate with respect  to a matter in the State List in the national interest \\ \midrule
  262 & Adjudication of disputes relating to waters of inter-state rivers or river valleys \\ \midrule
  263 & Provisions with respect to an inter-state council \\ \midrule
  265 & Taxes not to be imposed save by authority of law \\ \midrule
  275 & Grants from the Union to certain states \\ \midrule
  280 & Finance Commission \\ \midrule
  300 & Suits and proceedings \\ \midrule
  300 & A Persons not to be deprived of property save by authority of law (Right to property) \\ \midrule
  311 & Dismissal, removal or reduction in rank of persons employed in civil capacities under the Union or a state. \\ \midrule
  312 & All-India Services \\ \midrule
  315 & Public service commissions for the Union and for the states \\ \midrule
  320 & Functions of Public service commissions \\ \midrule
  323-A & Administrative tribunals \\ \midrule
  324 & Superintendence, direction and control of elections to be vested in an Election Commission \\ \midrule
  330 & Reservation of seats for scheduled castes and scheduled tribes in the House of the People \\ \midrule
  335 & Claims of scheduled castes and scheduled tribes to services and posts \\ \midrule
  352 & Proclamation of Emergency (National Emergency) \\ \midrule
  356 & Provisions in case of failure of constitutional machineryin states (President’s Rule) \\ \midrule
  360 & Provisions as to financial emergency. \\ \midrule
  365 & Effect of failure to comply with, or to give effect to, directions given by the Union (President’s Rule) \\ \midrule
  368 & Power of Parliament to amend the Constitution and procedure therefor \\ \midrule
  370 & Temporary provisions with respect to the state of  Jammu and Kashmir\\ \midrule
\end{longtable}

\theendnotes
\cleardoublepage