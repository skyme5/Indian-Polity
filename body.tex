% From File: B:/Writing/Books/Indian_Polity_V2/TeX_files/01_00.tex
%

\newgeometry{top=1in,bottom=1in,right=1in,left=1in}
\thispagestyle{empty}
\part{Constitutional Framework}
\restoregeometry
\cleardoublepage

% From File: B:/Writing/Books/Indian_Polity_V2/TeX_files/01_01.tex
%

. \gls{person:LORD-MOUNTBETTEN} became the first governor-general of the new Dominion of India. He swore in \gls{person:JAWAHARLAL-NEHRU} as the first prime minister of independent India. The Constituent Assembly of India formed in 1946 became the Parliament of the Indian Dominion.

\onecolumn

\begin{longtable}[c]{@{}|p{1cm}|p{4cm}|p{7cm}|@{}}
  \caption{Interim Government (1946)}
  \label{tbl:InterimGovernment}\\
  \toprule
  Sl. No. & Members & Portfolios Held \\* \midrule
  \endfirsthead
  %
  \multicolumn{3}{c}%
  {{\bfseries Table \thetable\ continued from previous page}} \\
  \toprule
  Sl. No. & Members & Portfolios Held \\* \midrule
  \endhead
  %
  \bottomrule
  \endfoot
  %
  \endlastfoot
  %
  1. & \gls{person:JAWAHARLAL-NEHRU} & External Affairs \& Commonwealth \\
     &                           & Relations \\
  2. & \gls{person:SARDAR-VALLABHBHAI-PATEL} & Home, Information \& Broadcasting \\
  3. & \gls{person:DR-RAJENDRA-PRASAD} & Food \& Agriculture \\
  4. & \gls{person:DR-JOHN-MATHAI} & Industries \& Supplies \\
  5. & \gls{person:JAGJIVAN-RAM} & Labor \\
  6. & \gls{person:SARDAR-BALDEV-SINGH} & Defense \\
  7. & \gls{person:C-H-BHABHA} & Works, Mines \& Power \\
  8. & \gls{person:LIAQUAT-ALI-KHAN} & Finance \\
  9. & \gls{person:ABDUR-RAB-NISHTAR} & Posts \& Air \\
  10. & \gls{person:ASAF-ALI} & Railways \& Transport \\
  11. & \gls{person:C-RAJAGOPALACHARI} & Education \& Arts \\
  12. & \gls{person:I-I-CHUNDRIGAR} & Commerce \\
  13. & \gls{person:GHAZNAFAR-ALI-KHAN} & Health \\
  14. & \gls{person:JOGINDER-NATH-MANDAL} & Law \\* \bottomrule
\end{longtable}
\textit{\textbf{Note}}: The members of the interim government were members of the Viceroy's Executive Council. The Viceroy continues to be the head of the Council. But, \gls{person:JAWAHARLAL-NEHRU} was designated as the Vice-President of the Council.

\begin{longtable}[c]{@{}|p{1cm}|p{4cm}|p{7cm}|@{}}
	\caption{First Cabinet of Free India (1947)}
	\label{tbl:firstCabinetOfFreeIndia}\\
	\toprule
	Sl. No. & Members & Portfolios Held \\* \midrule
	\endfirsthead
	%
	\multicolumn{3}{c}%
	{{\bfseries Table \thetable\ continued from previous page}} \\
	\toprule
	Sl. No. & Members & Portfolios Held \\* \midrule
	\endhead
	%
	\bottomrule
	\endfoot
	%
	\endlastfoot
	%
	1. & \gls{person:JAWAHARLAL-NEHRU} & Prime Minister; External Affairs \& Commonwealth \\
	&                           & Relations; Scientific Research \\
	2. & \gls{person:SARDAR-VALLABHBHAI-PATEL} & Home, Information \& Broadcasting; States \\
	3. & \gls{person:DR-RAJENDRA-PRASAD} & Food \& Agriculture \\
	4. & \gls{person:MAULANA-ABUL-KALAM-AZAD} & Education \\
	5. & \gls{person:DR-JOHN-MATHAI} & Railway \& Transport \\
	6. & \gls{person:R-K-SHANMUGHAM} & Finance \\
	7. & \gls{person:DR-B-R-AMBEDKAR} & Law \\
	8. & \gls{person:JAGJIVAN-RAM} & Labor \\
	9. & \gls{person:SARDAR-BALDEV-SINGH} & Defense \\
	10. & \gls{person:RAJ-KUMARI-AMRIT-KAUR} & Health \\
	11. & \gls{person:C-H-BHABHA} & Commerce \\
	12. & \gls{person:RAFI-AHMED-KIDWAI} & Communication \\
	13. & \gls{person:DR-SHYAMA-PRASAD-MUKHERJI} & Industries \& Supplies \\
	14. & \gls{person:V-N-GADGIL} & Works, Mines \& Power \\* \bottomrule
\end{longtable}

\theendnotes
\cleardoublepage

% From File: B:/Writing/Books/Indian_Polity_V2/TeX_files/01_02.tex
%

.
  \item \textbf{Lawyer–Politician Domination}: It is also maintained by the critics that the Constituent Assembly was dominated by lawyers and politicians. They pointed out that other sections of the society were not sufficiently represented. This, to them, is the main reason for the bulkiness and complicated language of the Constitution.
  \item \textbf{Dominated by Hindus}: According to some critics, the Constituent Assembly was a Hindu dominated body. \gls{person:LORD-VISCOUNT-SIMON} called it `a body of Hindus'. Similarly, \gls{person:WINSTON-CHURCHILL} commented that the Constituent Assembly represented `only one major community in India'.
\end{enumerate}


\section{Important Facts}

\begin{enumerate}
  \item Elephant was adopted as the symbol (seal) of the Constituent Assembly.
  \item \gls{person:SIR-B-N-RAU} was appointed as the constitutional advisor (Legal advisor) to the Constituent Assembly.
  \item \gls{person:H-V-R-IYENGAR} was the Secretary to the Constituent Assembly.
  \item \gls{person:S-N-MUKERJEE} was the chief draftsman of the constitution in the Constituent Assembly.
  \item \gls{person:PREM-BEHARI-NARAIN-RAIZADA} was the calligrapher of the Indian Constitution. The original constitution was handwritten by him in a flowing italic style.
  \item The original version was beautified and decorated by artists from Shantiniketan including \gls{person:NAND-LAL-BOSE} and \gls{person:BEOHAR-RAMMANOHAR-SINHA}.
  \item \gls{person:BEOHAR-RAMMANOHAR-SINHA} illuminated, beautified and ornamented the original Preamble calligraph-ed by \gls{person:PREM-BEHARI-NARAIN-RAIZADA}.
  \item The calligraphy of the Hindi version of the original constitution was done by \gls{person:VASANT-KRISHAN-VAIDYA} and elegantly decorated and illuminated by \gls{person:NAND-LAL-BOSE}.
\end{enumerate}


\onecolumn

%January \d+, \d+|February \d+, \d+|March \d+, \d+|April \d+, \d+|May \d+, \d+|June \d+, \d+|July \d+, \d+|August \d+, \d+|September \d+, \d+|October \d+, \d+|November \d+, \d+|December \d+, \d+

\begin{longtable}[c]{@{}|p{1cm}|p{5.5cm}|p{5.5cm}|@{}}
  \caption{Allocation of seats in the Constituent Assembly of India (1946)}
  \label{tab:AllocationSeatsConstituentAssembly1946}\\
  \toprule
  Sl.No. & Areas & Seats \\* \midrule
  \endfirsthead
  %
  \multicolumn{3}{c}%
  {{\bfseries Table \thetable\ continued from previous page}} \\
  \toprule
  Sl.No. & Areas & Seats \\* \midrule
  \endhead
  %
  \bottomrule
  \endfoot
  %
  \endlastfoot
  %
  1. & British Indian Provinces (11) & 292 \\
  2. & Princely States (Indian States) & 93 \\
  3. & Chief Commissioners’ Provinces (4) & 4 \\
  \toprule
  & Total & 389 \\* \bottomrule
\end{longtable}


\begin{longtable}[c]{@{}|p{1cm}|p{5.5cm}|p{5.5cm}|@{}}
  \caption{Results of the Elections to the Constituent Assembly (July–August 1946)}
  \label{tab:ResultsElectionsConstituentAssembly1946}\\
  \toprule
  Sl.No. & Name of the Party & Seats won \\* \midrule
  \endfirsthead
  %
  \multicolumn{3}{c}%
  {{\bfseries Table \thetable\ continued from previous page}} \\
  \toprule
  Sl.No. & Name of the Party & Seats won \\* \midrule
  \endhead
  %
  1. & Congress & 208 \\
  2. & Muslim League & 73 \\
  3. & Unionist Party & 1 \\
  4. & Unionist Muslims & 1 \\
  5. & Unionist Scheduled Castes & 1 \\
  6. & Krishak – Praja Party & 1 \\
  7. & Scheduled Castes Federation & 1 \\
  8. & Sikhs (Non-Congress) & 1 \\
  9. & Communist Party & 1 \\
  10. & Independents & 8 \\
  \toprule
  & Total & 296\\* \bottomrule
\end{longtable}

\begin{longtable}[c]{@{}|p{1cm}|p{5.5cm}|p{5.5cm}|@{}}
  \caption{Community-wise Representation in the Constituent Assembly (1946)}
  \label{tab:CommunityRepresentationConstituentAssembly1946}\\
  \toprule
  Sl.No. & Community & Strength \\* \midrule
  \endfirsthead
  %
  \multicolumn{3}{c}%
  {{\bfseries Table \thetable\ continued from previous page}} \\
  \toprule
  Sl.No. & Community & Strength \\* \midrule
  \endhead
  %
  1. & Hindus & 163 \\
  2. & Muslims & 80 \\
  3. & Scheduled Castes & 31 \\
  4. & Indian Christians & 6 \\
  5. & Backward Tribes & 6 \\
  6. & Sikhs & 4 \\
  7. & Anglo-Indians & 3 \\
  8. & Parsees & 3 \\
  \toprule
  Total & 296 & 1\\* \bottomrule
\end{longtable}

\begin{longtable}[c]{@{}|p{1cm}p{1cm}|p{6cm}|p{4cm}|@{}}
  \caption{State wise Membership of the Constituent Assembly of India as on 1947-12-31}
  \label{tab:StateWiseMemberShipAssembly}\\
  \toprule
  \multicolumn{2}{|c|}{S.No.} & Name & No. of Members \\
  \bottomrule
  \endfirsthead
  %
  \multicolumn{4}{c}%
  {{\bfseries Table \thetable\ continued from previous page}} \\
  \toprule
  \multicolumn{2}{|c}{S.No.} & Name & No. of Members \\
  \bottomrule
  \endhead
  %
  \multicolumn{2}{|c}{
  \textbf{A.}} & \textbf{Provinces (Indian Provinces) - 299} &  \\\bottomrule
  & 1. & Madras & 49 \\
  & 2. & Bombay & 21 \\
  & 3. & West Bengal & 19 \\
  & 4. & United Provinces & 55 \\
  & 5. & East Punjab & 12 \\
  & 6. & Bihar & 36 \\
  & 7. & C.P. and Berar & 17 \\
  & 8. & Assam & 8 \\
  & 9. & Orissa & 9 \\
  & 10. & Delhi & 1 \\
  & 11. & Ajmer-Merwara & 1 \\
  & 12. & Coorg & 1 \\
  \toprule
  \multicolumn{2}{|c}{\textbf{B.}} & \textbf{Indian States (Princely States) - 70} &  \\\bottomrule
  & 1. & Alwar & 1 \\
  & 2. & Baroda & 3 \\
  & 3. & Bhopal & 1 \\
  & 4. & Bikaner & 1 \\
  & 5. & Cochin & 1 \\
  & 6. & Gwalior & 4 \\
  & 7. & Indore & 1 \\
  & 8. & Jaipur & 3 \\
  & 9. & Jodhpur & 2 \\
  & 10. & Kolhapur & 1 \\
  & 11. & Kotah & 1 \\
  & 12. & Mayurbhanj & 1 \\
  & 13. & Mysore & 7 \\
  & 14. & Patiala & 2 \\
  & 15. & Rewa & 2 \\
  & 16. & Travancore & 6 \\
  & 17. & Udaipur & 2 \\
  & 18. & Sikkim and Cooch Behar Group & 1 \\
  & 19. & Tripura, Manipur and Khasi States Group & 1 \\
  & 20. & U.P. States Group & 1 \\
  & 21. & Eastern Rajputana States Group & 3 \\
  & 22. & Central India States Group & 3 \\
  & 23. & Western India States Group & 4 \\
  & 24. & Gujarat States Group & 2 \\
  & 25. & Deccan and Madras States Group & 2 \\
  & 26. & Punjab States Group & 3 \\
  & 27. & Eastern States Group I & 4 \\
  & 28. & Eastern States Group II & 3 \\
  & 29. & Residuary States Group & 4 \\
  \toprule
  &  & Total & 299\\* \bottomrule
\end{longtable}

\begin{longtable}[c]{@{}|p{6cm}|p{6cm}|@{}}
  \caption{Sessions of the Constituent Assembly at a Glance}
  \label{tab:SessionsConstituentAssembly}\\
  \toprule
  Sessions & Period \\
  \bottomrule
  \endfirsthead
  %
  \multicolumn{2}{c}%
  {{\bfseries Table \thetable\ continued from previous page}} \\
	\toprule
  Sessions & Period \\
	\midrule
  \endhead
  %
  First Session & 9–23 December, 1946 \\
  Second Session & 20–25 January, 1947 \\
  Third Session & 28 April–2 May, 1947 \\
  Fourth Session & 14–31 July, 1947 \\
  Fifth Session & 14–30 August, 1947 \\
  Sixth Session & 27 January, 1948 \\
  Seventh Session & 4 November, 1948 to 8 January, 1949 \\
  Eighth Session & 16 May–16 June, 1949 \\
  Ninth Session & 30 July–18 September, 1949 \\
  Tenth Session & 6–17 October, 1949 \\
  Eleventh Session & 14–26 November, 1949\\* \bottomrule
\end{longtable}

\theendnotes
\cleardoublepage
% From File: B:/Writing/Books/Indian_Polity_V2/TeX_files/01_03.tex
%


% From File: B:/Writing/Books/Indian_Polity_V2/TeX_files/01_04.tex
%


% From File: B:/Writing/Books/Indian_Polity_V2/TeX_files/02_00.tex
%


