% From File: B:/Writing/Books/Indian_Polity_V2/TeX_files/01_00.tex
%

\newgeometry{top=1in,bottom=1in,right=1in,left=1in}
\thispagestyle{empty}
\part{Constitutional Framework}
\restoregeometry
\cleardoublepage

% From File: B:/Writing/Books/Indian_Polity_V2/TeX_files/01_01.tex
%

\twocolumn

\chapter{Historical Background}

The British came to India in 1600 as traders, in the form of East India Company, which had the exclusive right of trading in India under a chapter granted by \gls{person:QUEEN-ELIZABETH-I}. In 1764, the Company, which till now had purely trading functions obtained the `diwani' (i.e., rights over revenue and civil justice) of Bengal, Bihar and Orissa\endnote{The Mughal Emperor, \gls{person:SHAH-ALAM}, granted ‘Diwani’ to the Company after its victory in the Battle of Buxar (1764).}. This started its career as a territorial power. In 1858-00-00, in the wake of the `Sepoy mutiny\index{default!Sepoy mutiny}', the British Crown assumed direct responsibility for the governance of India. This rule continued until India was granted independence on 1947-08-15.

With Independence came the need of a Constitution. As suggested by \gls{person:M-N-ROY} (a pioneer of communist movement in India and an advocate of Radical Democrat-ism) in 1934-00-00, a Constituent Assembly was formed for this purpose in 1946-00-00 and on 1950-01-26, the Constitution and polity have their roots in the British rule. There are certain events in the British rule that laid down the legal framework for the organization and functioning of government and administration in British India. These events have greatly influenced out contribution and polity. They are explained here in a chronological order:

\section{The Company Rule (1973-1858)}

\subsection{Regulating Act of 1773\index{Acts!Regulating Act of 1773}}
This act is of great constitutional importance as (a) it was the first step taken by the British Government to control and regulate the affairs of the East India Company in India; (b) it recognized, for the first time, the political and administrative functions of the Company; and (c) it laid the foundations of central administration in India.

\paragraph{Features of the Act}

\begin{enumerate}
  \item It designated the Governor of Bengal as the `Governor-General of Bengal' and created an Executive Council of four members to assist him. The first such Governor-General was Lord Warren Hastings.
  \item It made the governors of Bombay and Madras presidencies subordinate to the governor-general of Bengal, unlike earlier, when the three presidencies were independent of one another.
  \item It provided for the establishment of a Supreme Court at Calcutta (1774) comprising one chief justice and three other judges.
  \item It prohibited the servants of the Company from engaging in any private trade or accepting presents or bribes from the `natives'.
  \item It strengthened the control of the British Government over the Company by requiring the Court of Directors (governing body of the Company) to report on its revenue, civil, and military affairs in India.
\end{enumerate}

\subsection{Pitt's India Act of 1784\index{Acts!Pitt's India Act of 1784}}

In a bid to rectify the defects of the Regulating Act of 1773\index{Acts!Regulating Act of 1773}, the British Parliament passed the Amending Act of 1781\index{Acts!Amending Act of 1781}, also known as the Act of Settlement\index{Acts!Act of Settlement}. The next important act was the Pitt's India Act\endnote{It was introduced in the British Parliament by the then Prime Minister, \gls{person:WILLIAM-PITT}.} of 1784.

\paragraph{Features of the Act}
\begin{enumerate}
  \item It distinguished between the commercial and political functions of the company.
  \item It allowed the Court of Directors to manage the commercial affairs but created a new body called Board of Control to manage the political affairs. Thus, it established a system of double government.
  \item It employed the Board of Control to supervise and direct all operations of civil and military government or revenues of the British possessions in India.
\end{enumerate}

Thus, the act was significant for two reasons: first, the Company's territories in India were for the first time called the `British possessions in India'; and second, the British Government was given the supreme control over Company's affairs and its administration in India.

\subsection{Charter Act of 1833\index{Acts!Charter Act of 1833}}

This Act was the final step towards the centralization in British India.

\paragraph{Features of the Act}
\begin{enumerate}
  \item It made the Governor-General of Bengal as the Governor-General of India and vested in him all civil and military powers. Thus, the act created, for the first time, a Government of India having authority over the entire territorial area possessed by the British in India. \gls{person:LORD-WILLIAM-BENTICK} was the first governor-general of India.
  \item It deprived the governor of Bombay and Madras of their legislative powers. The Governor-General of India was given exclusive legislative powers for the entire British India. The law made under the previous acts were called as Regulation while laws made under this act were called as Acts.
  \item It ended the activities of the East India Company as a commercial body, which became a purely administrative body. It provided that the company's territories in India were held by it `in trust for His Majesty, His heirs and successors'.
  \item The Charter Act of 1833\index{Acts!Charter Act of 1833} attempted to introduce a system of open competition for civil servants\index{default!civil servants}, and stated that the Indians should not be debarred from holding any place, office and employment under the Company. However, this provision negated after opposition from the Courts of Directors.
\end{enumerate}

\subsection{Charter Act of 1853\index{Acts!Charter Act of 1853}}

This was the last of the series of Charter Acts passed by the British Parliament between 1793 and 1853. It was a significant constitutional landmark.
\paragraph{Features of the Act}

\begin{enumerate}
  \item It separated, for the first time, the legislative and executive functions of the Governor-General's council. It provided for the addition of six new members called the legislative councilors to the council. In other words, it established a separate Governor-General's legislative council which came to be known as the Indian (Central) Legislative Council. This legislative wing of the council functioned as mini-Parliament, adopting the same procedures as the British Parliament. Thus, legislative, of the first time, was treated as a special function of the government, requiring special machinery and special process.
  \item It introduced an open competition system of selection and recruitment of civil servants\index{default!civil servants}\endnote{At that time, the Civil Services of the company were classified into covenanted civil services (higher civil services) and uncovenanted civil services (lower civil services). The former was created by a law of the Company, while the later was created otherwise.} was thus thrown open to Indians also. Accordingly, the Macaulay Committee (the Committee on the Indian Civil Services) was appointed in 1854.
  \item It extended the Company's rule and allowed it to retain the possession of Indian territories on trust for the British Crown. But, it did not specify any particular period, unlike the previous Charters. This was a clear indication that the Company' rule could be terminated at any time the Parliament liked.
  \item It introduced, for the first time, local representation in the Indian (Central) Legislative Council. Of the six new legislative members of the governor-general's council, four members were appointed by the local (provincial) governments of Madras, Bombay, Bengal and Agra.
\end{enumerate}

\section{The Crown Rule}

\subsection{Government of India Act of 1858\index{Acts!Government of India Act of 1858}}

This significant Act was enacted in the wake of the Revolt of 1857\index{default!Revolt of 1857} - also known as the First War of Independence of the `Sepoy mutiny\index{default!Sepoy mutiny}'. The act known as the \textbf{Act for the Good Government of India\index{Acts!Act for the Good Government of India}}, abolished the East India Company, and transferred the powers of government, territories and revenues to British Crown.

\paragraph{Features of the Act}
\begin{enumerate}
  \item It provided that India henceforth was to be governed by, and in the name of, Her Majesty. It changed the designation of th Governor-General of India to that of Viceroy of India. He (viceroy) was the direct representative of the British Crown in India. \gls{person:LORD-CANNING} thus became first Viceroy of India.
  \item It ended the system of double government by abolishing the Board of Control and Court of Directors.
  \item It created a new office, Secretary of State for India, vested with complete authority and control over Indian administration. The secretary of state was a member of the British cabinet and was responsible ultimately to the British Parliament.
  \item It established a 15-member Council of India to assist the secretary of state for India. The council was an advisory body. The secretary of state was made the chairman of the council.
  \item It constituted the secretary of state-in-council as a body corporate, capable of suing and being sued in India and in England.
\end{enumerate}

`The Act of 1858 was, however, largely confined to improvement of the administrative machinery by which the Indian Government was to be supervised and controlled in England. It did not alter in any substantial way the system of government that prevailed in India.\endnote{Subhash C. Kashyap, Our Constitution, National Book Trust, Third Edition, 2001, P. 14.}, National Book Trust, 3rd Edition, 2001, P. A-10}'


\subsection{Indian Councils Act of 1861, 1892 and 1909\index{Acts!Indian Councils Act of 1861, 1892 and 1909}}

After the great revolt of 1857, the British Government felt the necessity of seeking the cooperation of the Indians in the administration of their country. In pursaunce of this policy of association, three acts were enacted by the British Parliament in 1864, 1892 and 1909. The Indian Council Act of 1864 is an important landmark in the constitutional and political history of India.

\paragraph{Features of the Act of 1861}
\begin{enumerate}
  \item It made a beginning of representative institutions by association Indians with the law-making process. It thus provided that the viceroy should nominate some Indians as non-official members of his expanded council. In 1862, \gls{person:LORD-CANNING}, the then viceroy, nominated three Indians to his legislative council — the Raja of Benaras, the Maharaja of Patiala and \gls{person:DINKAR-RAO}Sir \gls{person:DINKAR-RAO}.
  \item It initiated the process of decentralization by restoring the legislative powers to the Bombay and Madras Presidencies. It thus reversed the centralizing tendency that started from the Regulating Act of 1773\index{Acts!Regulating Act of 1773} and reached its climax under the Charter Act of 1833\index{Acts!Charter Act of 1833}. This policy of legislative devolution resulted in the grant of almost complete internal autonomy to the provinces in 1937-00-00.
  \item It also provided for the establishment of new legislative councils for Bengal, North-Western Frontier Province (NWFP) and Panjab, which were established in 1862, 1866 and 1897 respectively.
  \item It empowered the Viceroy to make rules and orders for the more convenient transaction of business in the council. It also gave a recognition to the  `portfolio' system, introduced by \gls{person:LORD-CANNING}\gls{person:LORD-CANNING} in 1859. Under this, a member of the Viceroy's council was made in-charge of one or more departments of the government and was authorized to issue final orders on behalf of the council on matters of his department(s).
  \item It empowered the Viceroy to issue ordinance, without the concurrence of the legislative council, during an emergency. The life of such ordinance was six months.
\end{enumerate}

\paragraph{Features of the Act of 1892}
\begin{enumerate}
  \item It increased the number of additional (non-official) members in the Central and provincial legislative councils, but maintained the official majority in them.
  \item It increased the function of legislative councils and gave them the power of discussing the budget\endnote{The system of Budget was introduced in British India in 1860.} and addressing questions to executive.
  \item It provided for the nomination of some non-official members of the
  \begin{list}{}{}
    \item[(a)] Central Legislative Council by the viceroy on the recommendation of the provincial legislative councils and the Bengal Chamber of Commerce,
    \item[(b)] that of the Provincial legislative councils by the Governors on the recommendation of the district boards, municipalities, universities, trade associations, zamindars and chambers.
  \end{list}
\end{enumerate}
`The act made a limited and indirect provision for the use of election in filling up some of the non-official seats both in the Central and provincial legislative councils. The word ``election'' was, however not used in the act. The process was described as nomination made on the recommendation of certain bodies\endnote{V. N. Shukla, The Constitution of India, Eastern Book Company, Tenth Edition, 2001, P. A-10.}.'

\paragraph{Features of the Act of 1909}
This Act is also known as Morley-Minto Reforms (Lord Morley was the then Secretary of State for India and \gls{person:LORD-MINTO} was the then Viceroy of India).
\begin{enumerate}
  \item It considerably increased the size of the legislative councils, both Central and Provincial. The number of members in the Central Legislative Council was raised from 16 to 60. The number of members in the provincial legislative councils was not uniform.
  \item It retained official majority in the Central Legislative Council but allowed the provincial legislative councils to habe non-official majority.
  \item It enlarged the deliberative functions of the legislative councils at both the levels. For example, members were allowed to ask supplementary questions, move resolutions on the budget, and so on.
  \item It provided (for the first time) for the association of Indians with the executive Councils of the Viceroy and Governors. Satyendra Prasad Sinha became the first Indian to join the Viceroy's Executive Council. He was appointed as the law member.
  \item It introduced a system of communal representation\index{default!communal representation} for Muslims by accepting the concept of `Separate Electorate'. Under this, the Muslim members were to be elected only by Muslim voter. Thus, the Act `legalized communal-ism' and \gls{person:LORD-MINTO} came to be known as the Father of Communal Electorate.
  \item It also provided for the separate representation of presidency corporations, chambers of commerce, universities and zamindars.
\end{enumerate}

\subsection{Government of India Act of 1919\index{Acts!Government of India Act of 1919}}

On 1917-08-20, the British Government declared, for the first time, that its objective was the gradual introduction of responsible government in India\endnote{The declaration thus stated: ‘The policy of His Majesty’s Government is that of the increasing association of Indians in every branch of the administration, and the gradual development of self-government institutions, with a view to the progressive realisation of responsible government in India as an integral part of the British Empire’.}.

The Government of India Act of 1919\index{Acts!Government of India Act of 1919}-00-00 was thus enacted, which came into force in 1921-00-00. This Act is also known as \gls{person:MONTAGU}-\gls{person:CHELMSFORD} Reforms (\gls{person:MONTAGU} was the Secretary of State for India and \gls{person:CHELMSFORD} was the Viceroy of India).

\paragraph{Features of the Act}
\begin{enumerate}
  \item It relaxed the central control over the provinces by demarcating and separating the central and provincial subjects. The central and provincial legislatures were authorized to make laws on their respective list of subjects. However, the structure of government continued to be
  centralized and unitary.
  \item It further divided the provincial subjects into two parts - transferred and reserved. The transferred subjects were to be administered by the governor with the aid of ministers responsible to the legislative Council. The reserved subjects, on the other hand, were to be administered by the governor and his executive council without being responsible to the legislative Council. This dual scheme of governance was known as
   `dyarchy' - a term derived from the Greek word di-arche which means double rule. However, this experiment was largely unsuccessful.
  \item It introduced, for the first time, bicameralism\index{default!bicameralism} and direct elections in the country. Thus, the Indian Legislative Council was replaced by bicameral legislative consisting of an Upper House (Council of State) and a Lower House (Legislative Assembly). The majority of members of both the Houses were chosen by direct election.
  \item It required that the three of the six members of the Viceroy's executive council (other than the commander-in-chief) were to be Indian.
  \item It extended the principle of communal representation\index{default!communal representation} by providing separate electorate\index{default!separate electorate}s for Sikhs, Indian Christias, Anglo-Indians and Europeans.
  \item It granted franchise to a limited number of people on the basis of property, tax or education.
  \item It created a new office of the High Commissioner for India in London and transferred to him some of the functions hitherto performed by the Secretary of State for India.
  \item It provided for the establishment of a public service commission\index{default!public service commission}. Hence, a Central Public Service Commission was set up in 1926-00-00 for recruiting civil servants\index{default!civil servants}\endnote{This was done on the recommendation of the Lee Commission on Superior Civil Services in India (1923–24).}.
  \item It separated, for the first time, provincial budgets from the Central budget and authorized the provincial legislatures to enact their budgets.
  \item It provided for the appointment of a statutory commission to inquire into and report on its working after ten years of its coming into force.
\end{enumerate}

\paragraph{Simon Commission\index{default!Simon Commission}}

In 1927-11-00, itself (i.e. 2 years before the schedule), the British Government announced the appointment of a seven member statutory commission under the chairmanship of \gls{person:SIR-JOHN-SIMON} to report on the condition of India under its new Constitution. All the members of the commission were British and hence, all the parties boycotted the commission, The commission submitted its report in 1930 and recommended the abolition of dyarchy, extension of responsible government in the provinces, establishment of a federation of British India and princely states, continuation of communal electorate and so on. To consider the proposal of the commission, the British Government convened three round table conferences of the representatives of the British Government, British India and Indian princely state. On the basis of these discussions, a `White Paper on Constitutional Reforms' was prepared and submitted for the consideration of the Joint Select Committee of the British Parliament. The recommendations of this committee were incorporated (with certain changes) in the next Government of India Act of 1935\index{Acts!Government of India Act of 1935}.

\paragraph{Communal Award\index{default!Communal Award}}
In 1932-08-00, \gls{person:RAMSAY-MACDONALD}, the British Prime Minister, announced a scheme of representation of the minorities, which came to be known as the Communal Award\index{default!Communal Award}. The award not only continued separate electorate\index{default!separate electorate}s for the Muslims, Sikhs, Indian Christians, Anglo-Indians and Europeans but also extended it to the depressed classes (schedules castes). \gls{person:MAHATMA-GANDHI} was distressed over this extension of the principle of communal representation\index{default!communal representation} to the depressed classes and undertook fast unto death in Yervada Jail\index{default!Yervada Jail} (Poona) to get the award modified. At last, there was an agreement between the leaders of the Congress and the depressed classes. The agreement, known and Poona Pact\index{default!Poona Pact}, retained the Hindu joint electorate and gave reserved seats to the depressed classes.



\subsection{Government of India Act of 1935\index{Acts!Government of India Act of 1935}}

The Act marked a second milestone towards a completely responsible government in India. It was a lengthy and detailed document having 321 Sections and 10 Schedules.

\paragraph{Features of the Act}
\begin{enumerate}
  \item It provided for the establishment of an All-India Federation consisting of provinces and princely states as units. The Act divided the powers between the Centre and units in terms of three lists - Federal List (for Center, with 59 items), Provincial List (for provinces, with 54 items) and the Concurrent List (for both, with 36 items). Residuary powers were given to the Viceroy. However, the federation never came into being as the princely states did not join it.
  \item It abolished dyarchy in the provinces and introduced `provincial autonomy' in its place. The provinces were allowed to act as autonomous units of administration in their defined spheres. Moreover, the Act introduced responsible governments in provinces, that is, the governor was required to act with the advice of ministers responsible to provincial legislature. This came into effect in 1937 and was discontinued in 1939.
  \item It provided for the adoption of dyarchy at the Centre. Consequently, the federal subjects were divided into reserved subjects and transferred subjects. However, this provision of the Act did not come into operation at all.
  \item It introduced bicameralism\index{default!bicameralism} in six out of eleven provinces. Thus, the legislature of Bengal, Bombay, Madras, Bihar, Assam and the United Provinces were made bicameral consisting of a legislative council (upper house) and a legislative assembly (lower house). However, many restrictions were placed on them.
  \item It further extended the principle of communal representation\index{default!communal representation} by providing separate electorate\index{default!separate electorate}s for depressed classes (schedules castes), women and labor (workers).
  \item It abolished the Council of India, established by the Government of India Act of 1858\index{Acts!Government of India Act of 1858}. The secretary of state for India was provided with a team of advisors.
  \item It extended franchise. About 10 percent of the total population got the voting right.
  \item It provided for the establishment of Reserve Bank of India\index{default!Reserve Bank of India} to control the currency and credit of the country.
  \item It provided for the establishment of not only Federal Public Service Commission but also a Provincial Public Service Commission and Joint Public Service Commission for two or more provinces.
  \item It provided for the establishment of a Federal Court, which was set up in 1937.
\end{enumerate}


\subsection{Indian Independence Act of 1947\index{Acts!Indian Independence Act of 1947}}

On 1947-02-20, the British Prime Minister \gls{person:CLEMENT-ATLEE} declared that the British rule in India would end by 1948-06-20; after which the power would be transferred to responsible Indian hands. This announcement was followed by the agitation by the Muslim League demanding partition of the country. Again on 1947-06-03 the British Government made it clear that any Constitution framed by the Constituent Assembly of India (formed in 1946) cannot apply to those parts of the country which were unwilling to accept it. On the same day (1947-06-03), \gls{person:LORD-MOUNTBETTEN}, the viceroy of India, put forth the partition plan known as the Mountbatten Plan\index{default!Mountbatten Plan}. The plan was accepted by the Congress and the Muslim League. Immediate effect was given to the plan by enacting the Indian Independence Act\endnote{The Indian Independence Bill was introduced in the British Parliament on July 4, 1947 and received the Royal Assent on July 18, 1947. The act came into force on August 15, 1947.} (1947).

\paragraph{Features of the Act}
\begin{enumerate}
  \item It ended the British rule in India and declared India as an independent and sovereign state from 1947-08-15.
  \item It provided for the partition of India and creation of two independent dominions of India and Pakistan with the right to secede from the British Commonwealth\index{default!British Commonwealth}.
  \item It abolished the office of viceroy and provided, for each dominion, a governor-general, who was to be appointed by the British King on the advice of the dominion cabinet. His Majesty's Government in Britain was to have no responsibility with respect to the Government of India or Pakistan.
  \item It empowered the Constituent Assemblies of both the dominions to legislate for their respective territories till the new constitutions were drafted and enforced. No Act of the British Parliament passed after 15, 1947 was to extend to either of the new dominion unless it was extended thereto by a law of the legislature of the dominion.
  \item It abolished the office of the secretary of state for India and transferred his function to the secretary of state for Commonwealth Affairs.
  \item It proclaimed the lapse of British paramountcy over the Indian princely states and treaty relations with tribal areas from 1947-08-15.
  \item It granted freedom to the Indian princely state either to join the Dominion of India or Dominion of Pakistan or to remai independent.
  \item It provided for the Governance of each of the dominions and the provinces by the Government of India Act of 1935\index{Acts!Government of India Act of 1935}, till the new Constitutions were framed. The dominions were however authorized to make modifications in the Act.
  \item It deprived the British Monarch of his right to veto bills or ask for reservation of certain bills for his approval. But, this right was reserved for the Governor-General. The Governor-General would have full power to assent to any bill in the name of His Majesty.
  \item It designated the Governor-General of Indian and the provincial governors as constitutional (nominal) heads of the states. They were made to act on the advice of the respective council of ministers in all matters.
  \item It discontinued the appointment to civil services and reservation of posts by the secretary of state for India. The members of the civil services appointed before 1947-08-15, would continue to enjoy all benefits that they were entitled to till that time.
\end{enumerate}

At the stroke of midnight 14-15 August, 1947, the British rule came to an end and power was transferred to the two new independent Dominions of India and Pakistan\endnote{The boundaries between the two Dominions were determined by a Boundary Commission headed by Radcliff. Pakistan included the provinces of West Punjab, Sind, Baluchistan, East Bengal, North-Western Frontier Province and the district of Sylhet in Assam. The referendum in the North-Western Frontier Province and Sylhet was in favour of Pakistan.}. \gls{person:LORD-MOUNTBETTEN} became the first governor-general of the new Dominion of India. He swore in \gls{person:JAWAHARLAL-NEHRU} as the first prime minister of independent India. The Constituent Assembly of India formed in 1946 became the Parliament of the Indian Dominion.

\onecolumn

\begin{longtable}[c]{@{}|p{1cm}|p{4cm}|p{7cm}|@{}}
  \caption{Interim Government (1946)}
  \label{tbl:InterimGovernment}\\
  \toprule
  Sl. No. & Members & Portfolios Held \\* \midrule
  \endfirsthead
  %
  \multicolumn{3}{c}%
  {{\bfseries Table \thetable\ continued from previous page}} \\
  \toprule
  Sl. No. & Members & Portfolios Held \\* \midrule
  \endhead
  %
  \bottomrule
  \endfoot
  %
  \endlastfoot
  %
  1. & \gls{person:JAWAHARLAL-NEHRU} & External Affairs \& Commonwealth \\
     &                           & Relations \\
  2. & \gls{person:SARDAR-VALLABHBHAI-PATEL} & Home, Information \& Broadcasting \\
  3. & \gls{person:DR-RAJENDRA-PRASAD} & Food \& Agriculture \\
  4. & \gls{person:DR-JOHN-MATHAI} & Industries \& Supplies \\
  5. & \gls{person:JAGJIVAN-RAM} & Labor \\
  6. & \gls{person:SARDAR-BALDEV-SINGH} & Defense \\
  7. & \gls{person:C-H-BHABHA} & Works, Mines \& Power \\
  8. & \gls{person:LIAQUAT-ALI-KHAN} & Finance \\
  9. & \gls{person:ABDUR-RAB-NISHTAR} & Posts \& Air \\
  10. & \gls{person:ASAF-ALI} & Railways \& Transport \\
  11. & \gls{person:C-RAJAGOPALACHARI} & Education \& Arts \\
  12. & \gls{person:I-I-CHUNDRIGAR} & Commerce \\
  13. & \gls{person:GHAZNAFAR-ALI-KHAN} & Health \\
  14. & \gls{person:JOGINDER-NATH-MANDAL} & Law \\* \bottomrule
\end{longtable}
\textit{\textbf{Note}}: The members of the interim government were members of the Viceroy's Executive Council. The Viceroy continues to be the head of the Council. But, \gls{person:JAWAHARLAL-NEHRU} was designated as the Vice-President of the Council.

\begin{longtable}[c]{@{}|p{1cm}|p{4cm}|p{7cm}|@{}}
	\caption{First Cabinet of Free India (1947)}
	\label{tbl:firstCabinetOfFreeIndia}\\
	\toprule
	Sl. No. & Members & Portfolios Held \\* \midrule
	\endfirsthead
	%
	\multicolumn{3}{c}%
	{{\bfseries Table \thetable\ continued from previous page}} \\
	\toprule
	Sl. No. & Members & Portfolios Held \\* \midrule
	\endhead
	%
	\bottomrule
	\endfoot
	%
	\endlastfoot
	%
	1. & \gls{person:JAWAHARLAL-NEHRU} & Prime Minister; External Affairs \& Commonwealth \\
	&                           & Relations; Scientific Research \\
	2. & \gls{person:SARDAR-VALLABHBHAI-PATEL} & Home, Information \& Broadcasting; States \\
	3. & \gls{person:DR-RAJENDRA-PRASAD} & Food \& Agriculture \\
	4. & \gls{person:MAULANA-ABUL-KALAM-AZAD} & Education \\
	5. & \gls{person:DR-JOHN-MATHAI} & Railway \& Transport \\
	6. & \gls{person:R-K-SHANMUGHAM} & Finance \\
	7. & \gls{person:DR-B-R-AMBEDKAR} & Law \\
	8. & \gls{person:JAGJIVAN-RAM} & Labor \\
	9. & \gls{person:SARDAR-BALDEV-SINGH} & Defense \\
	10. & \gls{person:RAJ-KUMARI-AMRIT-KAUR} & Health \\
	11. & \gls{person:C-H-BHABHA} & Commerce \\
	12. & \gls{person:RAFI-AHMED-KIDWAI} & Communication \\
	13. & \gls{person:DR-SHYAMA-PRASAD-MUKHERJI} & Industries \& Supplies \\
	14. & \gls{person:V-N-GADGIL} & Works, Mines \& Power \\* \bottomrule
\end{longtable}

\theendnotes
\cleardoublepage
\endnote{}
% From File: B:/Writing/Books/Indian_Polity_V2/TeX_files/01_02.tex
%

\twocolumn

\chapter{Making of the Constitution}

\section{Demand For a Constituent Assembly}

It was in 1934-00-00 that the idea of a Constituent Assembly for India was put forward for the first time by \gls{person:M-N-ROY}, a pioneer of communist movement in India. In 1935-00-00, the Indian National Congress (INC), for the first time, officially demanded a Constituent Assembly to frame the Constitution of free India. In 1938-00-00, \gls{person:JAWAHARLAL-NEHRU}, on behalf of the INC declared that `the Constitution of free India must be framed, without outside interference, by a Constituent Assembly elected on the basis of adult franchise'.

The demand was finally accepted in principle by the British Government in what is known as the `August Offer' of 1940. In 1942-00-00, \gls{person:SIR-STAFFORD-CRIPPS}, a member of the cabinet, came India with a draft proposal of the British Government on the framing of an independent Constitution to be adopted after the World War II. The Cripps Proposals were rejected by the Muslim League which wanted India to be divide into two autonomous states with two separate Constituent Assemblies. Finally, a Cabinet Mission\endnote{The Cabinet Mission consisting of three members (Lord Pethick Lawrence, \gls{person:SIR-STAFFORD-CRIPPS} and A V Alexander) arrived in India on March 24, 1946. The Cabinet Mission published its plan on May 16, 1946.} was sent to India. While it rejected the idea of two Constituent Assemblies, it put forth a scheme for the Constituent Assembly which more or less satisfied the Muslim League.

\section{Composition of the Constituent Assembly}

The Constituent Assembly was constituted in 1946-11-00 under the scheme formulated by the Cabinet Mission Plan. The features of the scheme were

\paragraph{Features of the scheme}
\begin{enumerate}
  \item The total strength of the Constituent Assembly was to be 398. Of these, 296 seats were to be allotted to British India and 93 seats to the Princely States. Out of 296 seats allowed to the British India, 292 members were to be drawn from the eleven governors' provinces\endnote{These include Madras, Bombay, U P , Bihar, Central Provinces, Orissa, Punjab, NWFP, Sindh, Bengal and Assam.} and four from the four chied commissioners' provinces\endnote{These include Delhi, Ajmer–Merwara, Coorg and British Baluchistan.}, one from each.
  \item Each province and princely state (or group of states in case of small state) were to be allotted seats in proportion to their respective population. Roughly, one seat was to be allotted for every million population.
  \item Seats allocation to each British province were to be divide among the three principle communities - Muslims, Sikhs and general (all except Muslims and Sikhs), in proportion to their population.
  \item The representatives of each community were to be elected by members of that community in the provincial legislative assembly and voting was to be by the method of proportional representation by the means of single transferable vote.
  \item The representatives of princely states were to be nominated by the heads of the princely states.
\end{enumerate}

It is thus clear that the Constituent Assembly was to be a partly elected and partly nominated body. Moreover, the members were to be indirectly elected by the members of provincial assemblies, who themselves were elected on a limited franchise\endnote{The Government of India Act of 1935\index{Acts!Government of India Act of 1935} granted limited franchise on the basis of tax, property and education.}.

The election to the Constituent Assembly (for 296 seats allotted to the British Indian Provinces) were held in July-August 1946. The Indian National Congress won 208 seats, the Muslim Leagues 73 seats, and the small groups and independents got the remaining 15 seats. However, the 93 seats allotted to the princely states were not filled as they decided to stay away from the Constituent Assembly.

Although the Constituent Assembly was not directly elected by the people of India on the basis of adult franchise, the Assembly comprised representatives of all section of Indian Society - Hindus, Muslims, Sikhs, Parsis, Anglo-Indians, Indian Christians, SCs, STs including women of all these sections. The Assembly included all important personalities of India at that time, with the exception of \gls{person:MAHATMA-GANDHI}.

\section{Working of the Constituent Assembly}

The Constituent Assembly held its first meeting on 1946-12-09. The Muslim League boycotted the meeting and insisted on a separate state of Pakistan. The meeting was thus attended by only 211 members. {\colord{}\gls{person:DR-SACHCHIDANAND-SINHA}, the oldest member, was elected as the temporary President of the Assembly, following the French practice.}

Later, \gls{person:DR-RAJENDRA-PRASAD} was elected as the President of the Assembly. Similarly, both \gls{person:H-C-MUKHERJEE} and \gls{person:V-T-KRISHNAMACHI} were elected as the Vice-Presidents of the Assembly. In other words, the Assembly had two Vice-Presidents.

\paragraph{Objectives Resolution\index{default!Objectives Resolution}}

On 1946-12-13, \gls{person:JAWAHARLAL-NEHRU} moved the historic `Objectives Resolution\index{default!Objectives Resolution}' in the Assembly. It laid down the fundamentals and philosophy of the constitutional structure. It read:

\begin{enumerate}
  \item \textquotedblleft This Constituent Assembly declares its firm and solemn resolve to proclaim India as an Independent Sovereign Republic and to draw up for her future governance a Constitution:
  \item Wherein the territories that now comprise British India, the territories that now from the Indian States, and such other territories as are willing to be consituted into the independent sovereign India, shall be a United of them all; and
  \item wherein the said territories, whether with their present boundaries or with such others as may be determined by the Constitution, shall possess and retain the status of autonomous units together with residuary powers and exercise all powers and functions of Government and administration save and except such powers and functions as are vested in or assigned to the Union or as are inherent or implied in the Union or resulting therefrom; and
  \item wherein all power and authority of the Sovereign Independent India, its constituent parts and organs of Government are derived from the people; and
  \item wherein shall be guaranteed and secured to all the people of India justice, social, economic and political; equality of status and of opportunity, and before the law; freedom of thought, expression, belied, faith, worship, vocation, association and action, subject to law and public morality; and
  \item wherein adequate safeguards shall be provided for minorities, backwards and tribal areas, and depressed and other backward classes; and
  \item whereby shall be maintained the integrity of the territory of the Republic and its sovereign rights on land, sea and air according to justice and the law of civilized nations; and
  \item This ancient land attains its rightful and honored place in the world and makes its full and willing contribution to the promotions of world peace and the welfare of mankind.\textquotedblright
  
\end{enumerate}

This Resolution was unanimously adopted by the Assembly on 1947-01-22. It influenced the eventual shaping of the constitution through all its subsequent stages. Its modified version forms the Preamble of the present Constitution.

\section{Changes by the Independence Act}

The representatives of the princely states, who had stayed awat from the Constituent Assembly, gradually joined in. On 1947-04-28, representatives of the six states\endnote{These include Baroda, Bikaner, Jaipur, Patiala, Rewa and Udaipur.} were part of the Assembly. After the acceptance of Mountbatten Plan\index{default!Mountbatten Plan} of 1947-06-03 for a partition of the country, the representatives of most of the other princely states took their seats in the Assembly. The members of the Muslim League from the Indian Dominion also entered the Assembly.

The Indian Independence Act of 1947\index{Acts!Indian Independence Act of 1947} made the following three changes in the position of the Assembly:

\begin{enumerate}
  \item The Assembly was made a fully sovereign body, which could frame any Constitution it pleased. The act empowered the Assembly to abrogate or alter any law made by the British Parliament in relation to India.
  \item The Assembly also became a legislative body. In other words, two separate functions were assigned to the Assembly, that is, making of a constitution for free India and enacting of ordinary laws for the country. These two tasks were to be performed on separate days. Thus, the Assembly became the first Parliament of free India (Dominion Legislature). Whenever the Assembly met as the Constituent body it was chaired by \gls{person:DR-RAJENDRA-PRASAD} and when it met as the legislative body\endnote{For the first time, the Constituent Assembly met as Dominion Legislature on November 17, 1947 and elected G V Mavlankar as its speaker.}, it was chaired by \gls{person:G-V-MAVALANKAR}. These two functions continued till 1949-11-26, when the task of making the Constitution was over.
  \item The Muslim League members (hailing from the areas\endnote{These are West Punjab, East Bengal, NWFP, Sindh, Baluchistan and Sylhet District of Assam. A separate Constituent Assembly was set up for Pakistan.} included in the Pakistan) withdrew from the Constituent Assembly for India. Consequently, the total strength of the Assembly came down to 299 as against 389 originally fixed in 1946 under the Cabinet Mission Plan. The strength of the Indian provinces (formerly British Provinces) was reduced from 296 to 299 and those of the princely states from 93 to 70. The state-wise membership of the Assembly as on 1947-12-31, is shown in \ref{tab:StateWiseMemberShipAssembly} at the end of this chapter.
\end{enumerate}

\section{Other Functions Performed}

In addition to the making of the Constitution and enacting of ordinary laws, the Constituent Assembly also performed the following functions:

\begin{enumerate}
  \item It ratified the India's membership of the Commonwealth in 1949-05-00.
  \item It adopted the national flag on 1950-01-24.
  \item It adopted the national anthem on 1950-01-24.
  \item It adopted the national song on 1950-01-24.
  \item It elected \gls{person:DR-RAJENDRA-PRASAD} as the first President of India on 1950-01-24.
\end{enumerate}

In all, the Constituent Assembly had 11 sessions over two years, 11 month and 18 days. The Constitution-makes had gone through the constitutions of about 60 countries, and the Draft Constitution was considered for 114 days. The total expenditure incurred on making the Constitution amounted to 64 lakh.

On 1950-01-24, the Constituent Assembly held its final session. It however, did not end, and continued as the provisional parliament of India from 1950-01-26 till the formation of new Parliament\endnote{The Provisional Parliament ceased to exist on April 17, 1952. The first elected Parliament with the two Houses came into being in May 1952.} after the first general elections in 1951-52.

\section{Committees of the Constituent Assembly}

The Constituent Assembly appointed a number of committees to deal with different tasks of constitution-making. Out of these, eight were major committees and the others were minor committees. The names of the committees and their chairman are given below:

\subsection{Major Committees}

\begin{enumerate}
  \item \gls{committee:UNION-POWERS-COMMITTEE} - \gls{person:JAWAHARLAL-NEHRU}
  \item \gls{committee:UNION-CONSTITUTION-COMMITTEE} - \gls{person:JAWAHARLAL-NEHRU}
  \item \gls{committee:PROVINCIAL-CONSTITUTION-COMMITTEE} - \gls{person:SARDAR-VALLABHBHAI-PATEL}
  \item \gls{committee:DRAFTING-COMMITTEE} - \gls{person:DR-B-R-AMBEDKAR}
  \item \gls{committee:ADVISORY-COMMITTEE} on Fundamental Rights\index{default!Fundamental Rights}, Minorities and Tribal and Excluded Areas - \gls{person:SARDAR-VALLABHBHAI-PATEL}. This committee had the following five sub-committees:
  \begin{enumerate}
    \item Fundamental Rights\index{default!Fundamental Rights} Sub-Committee - \gls{person:J-B-KRIPALANI}
    \item \gls{committee:MINORITIES-SUB-COMMITTEE} - \gls{person:H-C-MUKHERJEE}
    \item \gls{committee:NORTH-EAST-FRONTIER-TRIBUNAL-AREAS-SUB-COMMITTEE} (with Assam Excluded \& Partially Excluded Areas) - \gls{person:GOPINATH-BARDOLOI}
    \item \gls{committee:EXCLUDED-AND-PARTIALLY-EXCLUDED-AREAS} Other than those in Assam Sub Committee - \gls{person:A-V-THAKKAR}
    \item \gls{committee:NORTH-WEST-FRONTIER-TRIBAL-AREAS-SUB-COMMITTEE}\endnote{One of the political consequences of the British Government’s statement of June 3, 1947, was that following a referendum, the North-West Frontier Province and Baluchistan became part of the territory of the Dominion of Pakistan and as a result the tribal areas in this region became a concern of that Dominion. The Sub-Committee on the Tribal Areas in the North-West Frontier Province and Baluchistan was not therefore called upon to function on behalf of the Constituent Assembly of India. (B. Shiva Rao, The Framing of India’s Constitution : Select Documents, Volume III, P.681.) The members of this Sub-Committee were : Khan Abdul Ghaffar Khan, Khan Abdul Samad Khan and Mehr Chand Khanna. The information about the Chairman is not found.}
  \end{enumerate}
  
  \item \gls{committee:RULES-OF-PROCEDURE-COMMITTEE} - \gls{person:DR-RAJENDRA-PRASAD}
  \item \gls{committee:STATES-COMMITTEE-COMMITTEE-FOR-NEGOTIATING-WITH-STATES-} - \gls{person:JAWAHARLAL-NEHRU}
  \item \gls{committee:STEERING-COMMITTEE} - \gls{person:DR-RAJENDRA-PRASAD}
\end{enumerate}

\subsection{Minor Committees}

\begin{enumerate}
  \item \gls{committee:FINANCE-AND-STAFF-COMMITTEE} - \gls{person:DR-RAJENDRA-PRASAD}
  \item \gls{committee:CREDENTIALS-COMMITTEE} - \gls{person:ALLADI-KRISHNASWAMI-AYYAR}
  \item \gls{committee:HOUSE-COMMITTEE} - \gls{person:B-PATTABHI-SITARAMAYYA}
  \item \gls{committee:ORDER-OF-BUSINESS-COMMITTEE} - \gls{person:DR-K-M-MUNSHI}
  \item \gls{committee:AD-HOC-COMMITTEE-ON-THE-NATIONAL-FLAG} - \gls{person:DR-RAJENDRA-PRASAD}
  \item \gls{committee:COMMITTEE-ON-THE-FUNCTIONS-OF-THE-CONSTITUENT-ASSEMBLY} - \gls{person:G-V-MAVALANKAR}
  \item \gls{committee:AD-HOC-COMMITTEE-ON-THE-SUPREME-COURT} - \gls{person:S-VARADACHARI} (Not an Assembly Member)
  \item \gls{committee:COMMITTEE-ON-CHIEF-COMMISSIONERS-PROVINCES} - \gls{person:B-PATTABHI-SITARAMAYYA}
  \item \gls{committee:EXPERT-COMMITTEE-ON-THE-FINANCIAL-PROVISIONS-OF-THE-UNION-CONSTITUTION} - \gls{person:NALINI-RANJAN-SARKAR} (Not an Assembly Member)
  \item \gls{committee:LINGUISTIC-PROVINCES-COMMISSION} - \gls{person:S-K-DAR} (Not an Assembly Member)
  \item \gls{committee:SPECIAL-COMMITTEE-TO-EXAMINE-THE-DRAFT-CONSTITUTION} - \gls{person:JAWAHARLAL-NEHRU}
  \item \gls{committee:PRESS-GALLERY-COMMITTEE} - \gls{person:USHA-NATH-SEN}
  \item \gls{committee:AD-HOC-COMMITTEE-ON-CITIZENSHIP} - \gls{person:S-VARADACHARI}
\end{enumerate}


\subsection{\gls{committee:DRAFTING-COMMITTEE}}

Among all the committees of the Constituent Assembly, the most important committee was the \gls{committee:DRAFTING-COMMITTEE} set up on 1947-08-29. It was this committee that was entrusted with the task of preparing a draft of the new Constitution. It consisted of seven members. They were

\begin{enumerate}
  \item \gls{person:DR-B-R-AMBEDKAR} ({\colord Chairmen})
  \item \gls{person:N-GOPALASWAMY-AYYANGAR}
  \item \gls{person:ALLADI-KRISHNASWAMI-AYYAR}
  \item \gls{person:DR-K-M-MUNSHI}
  \item \gls{person:SAYED-MOHAMMAD-SAADULLAH}
  \item \gls{person:N-MADHAVA-RAU} (He replaced \gls{person:B-L-MITTER} who resined due to ill-health)
  \item \gls{person:T-T-KRISHNAMACHARI} (He replaced \gls{person:D-P-KHAITAN} who died in 1948)
\end{enumerate}

The \gls{committee:DRAFTING-COMMITTEE}, after taking into consideration the proposal of the various committees, prepared the first draft of the Constitution of India, which was published in 1948-02-00. The people of India were given eight months to discuss the draft and propose amendments. In the light of the public comments, criticisms and suggestions, the \gls{committee:DRAFTING-COMMITTEE} prepared a second draft, which was published in 1948-10-00.

The \gls{committee:DRAFTING-COMMITTEE} took less than six months to prepare its draft. In all it say for 141 days.

\section{Enactment of the Constitution}

\gls{person:DR-B-R-AMBEDKAR} introduced the final draft of the Constitution in the Assembly on 1948-11-04 (first reading). The Assembly had a general discussion on it for five days (till 1948-11-09).

The second reading (clause by clause consideration) started on 1948-11-15 and ended on 1949-10-17. During this stage, as many as 7653 amendments were proposed and 2473 were actually discussed in the Assembly.

The third reading of the draft started on 1949-11-14. \gls{person:DR-B-R-AMBEDKAR} moved a motion - `the Constitution as settled by the Assembly be passed'. The motion on Draft Constitution was declared as passed on 1949-11-26, and received the signatures of the members and the president. Out of a total 299 members of the Assembly, only 284 were actually present on that day and signed the Constitution. This is also the date mentioned in the Preamble as the date on which the people of India in the Constituent Assembly adopted, enacted and gave to themselves this Constitution.

{\colord The Constitution as adopted on 1949-11-26, contained a Preamble, 395 Articles and 8 Schedules}. The Preamble was enacted after the entire Constitution was already enacted.

\gls{person:DR-B-R-AMBEDKAR}, the then Law Minister, piloted the Draft Constitution in the Assembly. He took a very prominent part in the deliberations of the Assembly. He was known for his logical, forceful and persuasive arguments on the floor of the Assembly. He is recognized as the `Father of the Constitution of India'. This brilliant writer, constitutional expert, undisputed leader of the scheduled castes and the `chief architech of the Constitution of India' is also known as a `Modern Manu'.

\section{Enforcement of the Constitution}

Some provisions of the Constitution pertaining to citizenship, elections, provisional parliament, temporary and transitional provisions, and short title contained in Articles 5, 6, 7, 8, 9, 60, 324, 366, 367, 379, 380, 388, 391, 392 and 393 came into force on 1949-11-26, itself.

The remaining provisions(the major part) of the Constitution came into force on 1950-01-26. This day is referred to in the Constitution as the `date of it commencement', and celebrated as the Republic Day.

January 26 was specifically chosen as the `date of commencement' of the Constitution because of its historical importance. It was on this day in 1930 that {\colord Purna Swaraj} day was celebrated, following the resolution of the Lahor Session (December 1929) of the INC.

With the commencement of the Constitution, the Indian Independence Act of 1947\index{Acts!Indian Independence Act of 1947} and the Government of India Act of 1935\index{Acts!Government of India Act of 1935}, with all enactments amending or supplementing the latter Act, were repealed. The Abolition of Privy Council Jurisdiction Act (1949) was however continued.

\section{Criticism of the Constituent Assembly}

The critics have criticized the Constituent Assembly on various grounds. These are as follows

\begin{enumerate}
  \item \textbf{Not a Representative Body}: The critics have argued that the Constituent  Assembly was not a representative body as its members were not directly elected by the people of India on the basis of universal adult franchise\index{default!universal adult franchise}.
  \item \textbf{Not a Sovereign Body}: The critics maintained that the Constituent Assembly was not a sovereign body as it was created by the proposals of the British Government. Further, they said that the Assembly held its sessions with the permission of the British Government.
  \item \textbf{Time Consuming}: According to the critics, the Constituent Assembly took unduly long time to make the Constitution. They stated that the framers of the American Constitution took only four months to complete their work. In this context, \gls{person:NAZIRUDDIN-AHMED}, a member of the Constituent Assembly, coined a new name for the \gls{committee:DRAFTING-COMMITTEE} to show his contempt for it. He called it a ``Drifting Committee''.
  \item \textbf{Dominated by Congress}: The critics charged that the Constituent Assembly was dominated by the Congress party. \gls{person:GRANVILLE-AUSTIN}, a British Constitutional expert, remarked: `The Constituent Assembly was a one-party body in an essentially one-party country. The Assembly was the Congress and the Congress was India'\endnote{\gls{person:GRANVILLE-AUSTIN}, The Indian Constitution—Cornerstone of a Nation, Oxford, 1966, P. 8.}.
  \item \textbf{Lawyer–Politician Domination}: It is also maintained by the critics that the Constituent Assembly was dominated by lawyers and politicians. They pointed out that other sections of the society were not sufficiently represented. This, to them, is the main reason for the bulkiness and complicated language of the Constitution.
  \item \textbf{Dominated by Hindus}: According to some critics, the Constituent Assembly was a Hindu dominated body. \gls{person:LORD-VISCOUNT-SIMON} called it `a body of Hindus'. Similarly, \gls{person:WINSTON-CHURCHILL} commented that the Constituent Assembly represented `only one major community in India'.
\end{enumerate}


\section{Important Facts}

\begin{enumerate}
  \item Elephant was adopted as the symbol (seal) of the Constituent Assembly.
  \item \gls{person:SIR-B-N-RAU} was appointed as the constitutional advisor (Legal advisor) to the Constituent Assembly.
  \item \gls{person:H-V-R-IYENGAR} was the Secretary to the Constituent Assembly.
  \item \gls{person:S-N-MUKERJEE} was the chief draftsman of the constitution in the Constituent Assembly.
  \item \gls{person:PREM-BEHARI-NARAIN-RAIZADA} was the calligrapher of the Indian Constitution. The original constitution was handwritten by him in a flowing italic style.
  \item The original version was beautified and decorated by artists from Shantiniketan including \gls{person:NAND-LAL-BOSE} and \gls{person:BEOHAR-RAMMANOHAR-SINHA}.
  \item \gls{person:BEOHAR-RAMMANOHAR-SINHA} illuminated, beautified and ornamented the original Preamble calligraph-ed by \gls{person:PREM-BEHARI-NARAIN-RAIZADA}.
  \item The calligraphy of the Hindi version of the original constitution was done by \gls{person:VASANT-KRISHAN-VAIDYA} and elegantly decorated and illuminated by \gls{person:NAND-LAL-BOSE}.
\end{enumerate}


\onecolumn

%January \d+, \d+|February \d+, \d+|March \d+, \d+|April \d+, \d+|May \d+, \d+|June \d+, \d+|July \d+, \d+|August \d+, \d+|September \d+, \d+|October \d+, \d+|November \d+, \d+|December \d+, \d+

\begin{longtable}[c]{@{}|p{1cm}|p{5.5cm}|p{5.5cm}|@{}}
  \caption{Allocation of seats in the Constituent Assembly of India (1946)}
  \label{tab:AllocationSeatsConstituentAssembly1946}\\
  \toprule
  Sl.No. & Areas & Seats \\* \midrule
  \endfirsthead
  %
  \multicolumn{3}{c}%
  {{\bfseries Table \thetable\ continued from previous page}} \\
  \toprule
  Sl.No. & Areas & Seats \\* \midrule
  \endhead
  %
  \bottomrule
  \endfoot
  %
  \endlastfoot
  %
  1. & British Indian Provinces (11) & 292 \\
  2. & Princely States (Indian States) & 93 \\
  3. & Chief Commissioners’ Provinces (4) & 4 \\
  \toprule
  & Total & 389 \\* \bottomrule
\end{longtable}


\begin{longtable}[c]{@{}|p{1cm}|p{5.5cm}|p{5.5cm}|@{}}
  \caption{Results of the Elections to the Constituent Assembly (July–August 1946)}
  \label{tab:ResultsElectionsConstituentAssembly1946}\\
  \toprule
  Sl.No. & Name of the Party & Seats won \\* \midrule
  \endfirsthead
  %
  \multicolumn{3}{c}%
  {{\bfseries Table \thetable\ continued from previous page}} \\
  \toprule
  Sl.No. & Name of the Party & Seats won \\* \midrule
  \endhead
  %
  1. & Congress & 208 \\
  2. & Muslim League & 73 \\
  3. & Unionist Party & 1 \\
  4. & Unionist Muslims & 1 \\
  5. & Unionist Scheduled Castes & 1 \\
  6. & Krishak – Praja Party & 1 \\
  7. & Scheduled Castes Federation & 1 \\
  8. & Sikhs (Non-Congress) & 1 \\
  9. & Communist Party & 1 \\
  10. & Independents & 8 \\
  \toprule
  & Total & 296\\* \bottomrule
\end{longtable}

\begin{longtable}[c]{@{}|p{1cm}|p{5.5cm}|p{5.5cm}|@{}}
  \caption{Community-wise Representation in the Constituent Assembly (1946)}
  \label{tab:CommunityRepresentationConstituentAssembly1946}\\
  \toprule
  Sl.No. & Community & Strength \\* \midrule
  \endfirsthead
  %
  \multicolumn{3}{c}%
  {{\bfseries Table \thetable\ continued from previous page}} \\
  \toprule
  Sl.No. & Community & Strength \\* \midrule
  \endhead
  %
  1. & Hindus & 163 \\
  2. & Muslims & 80 \\
  3. & Scheduled Castes & 31 \\
  4. & Indian Christians & 6 \\
  5. & Backward Tribes & 6 \\
  6. & Sikhs & 4 \\
  7. & Anglo-Indians & 3 \\
  8. & Parsees & 3 \\
  \toprule
  Total & 296 & 1\\* \bottomrule
\end{longtable}

\begin{longtable}[c]{@{}|p{1cm}p{1cm}|p{6cm}|p{4cm}|@{}}
  \caption{State wise Membership of the Constituent Assembly of India as on 1947-12-31}
  \label{tab:StateWiseMemberShipAssembly}\\
  \toprule
  \multicolumn{2}{|c|}{S.No.} & Name & No. of Members \\
  \bottomrule
  \endfirsthead
  %
  \multicolumn{4}{c}%
  {{\bfseries Table \thetable\ continued from previous page}} \\
  \toprule
  \multicolumn{2}{|c}{S.No.} & Name & No. of Members \\
  \bottomrule
  \endhead
  %
  \multicolumn{2}{|c}{
  \textbf{A.}} & \textbf{Provinces (Indian Provinces) - 299} &  \\\bottomrule
  & 1. & Madras & 49 \\
  & 2. & Bombay & 21 \\
  & 3. & West Bengal & 19 \\
  & 4. & United Provinces & 55 \\
  & 5. & East Punjab & 12 \\
  & 6. & Bihar & 36 \\
  & 7. & C.P. and Berar & 17 \\
  & 8. & Assam & 8 \\
  & 9. & Orissa & 9 \\
  & 10. & Delhi & 1 \\
  & 11. & Ajmer-Merwara & 1 \\
  & 12. & Coorg & 1 \\
  \toprule
  \multicolumn{2}{|c}{\textbf{B.}} & \textbf{Indian States (Princely States) - 70} &  \\\bottomrule
  & 1. & Alwar & 1 \\
  & 2. & Baroda & 3 \\
  & 3. & Bhopal & 1 \\
  & 4. & Bikaner & 1 \\
  & 5. & Cochin & 1 \\
  & 6. & Gwalior & 4 \\
  & 7. & Indore & 1 \\
  & 8. & Jaipur & 3 \\
  & 9. & Jodhpur & 2 \\
  & 10. & Kolhapur & 1 \\
  & 11. & Kotah & 1 \\
  & 12. & Mayurbhanj & 1 \\
  & 13. & Mysore & 7 \\
  & 14. & Patiala & 2 \\
  & 15. & Rewa & 2 \\
  & 16. & Travancore & 6 \\
  & 17. & Udaipur & 2 \\
  & 18. & Sikkim and Cooch Behar Group & 1 \\
  & 19. & Tripura, Manipur and Khasi States Group & 1 \\
  & 20. & U.P. States Group & 1 \\
  & 21. & Eastern Rajputana States Group & 3 \\
  & 22. & Central India States Group & 3 \\
  & 23. & Western India States Group & 4 \\
  & 24. & Gujarat States Group & 2 \\
  & 25. & Deccan and Madras States Group & 2 \\
  & 26. & Punjab States Group & 3 \\
  & 27. & Eastern States Group I & 4 \\
  & 28. & Eastern States Group II & 3 \\
  & 29. & Residuary States Group & 4 \\
  \toprule
  &  & Total & 299\\* \bottomrule
\end{longtable}

\begin{longtable}[c]{@{}|p{6cm}|p{6cm}|@{}}
  \caption{Sessions of the Constituent Assembly at a Glance}
  \label{tab:SessionsConstituentAssembly}\\
  \toprule
  Sessions & Period \\
  \bottomrule
  \endfirsthead
  %
  \multicolumn{2}{c}%
  {{\bfseries Table \thetable\ continued from previous page}} \\
	\toprule
  Sessions & Period \\
	\midrule
  \endhead
  %
  First Session & 9–23 December, 1946 \\
  Second Session & 20–25 January, 1947 \\
  Third Session & 28 April–2 May, 1947 \\
  Fourth Session & 14–31 July, 1947 \\
  Fifth Session & 14–30 August, 1947 \\
  Sixth Session & 27 January, 1948 \\
  Seventh Session & 4 November, 1948 to 8 January, 1949 \\
  Eighth Session & 16 May–16 June, 1949 \\
  Ninth Session & 30 July–18 September, 1949 \\
  Tenth Session & 6–17 October, 1949 \\
  Eleventh Session & 14–26 November, 1949\\* \bottomrule
\end{longtable}

\theendnotes
\cleardoublepage\endnote{}
% From File: B:/Writing/Books/Indian_Polity_V2/TeX_files/01_03.tex
%

\twocolumn

\chapter{Salient Features of the Constitution}

\section{Introduction}

The Indian Constitution is unique in its contents and spirit. Through borrowed from almost every constitution of the world, the constitution of India has several salient features that distinguish it from the constitution of other countries.

It should be notes al the outset that a number of original features of the Constitution (as adopted in 1949) have undergone a substantial change, on account of several amendments, particularly 7th, 42nd, 44th, 73rd, 74th and 97th Amendments. In fact, the 42nd Amendment Act of 1976, is known as `Mini-Constitution' due to the important and large number of changes made by it in various parts of the Constitution. However, in the {\colord \textit{\gls{person:KESHAVANANDA-BHARTI}}} case\endnote{Kesavananda Bharati v.State of Kerala, (1973)} (1973), the Supreme Court rules that the constitution power of Parliament under Article 368 does not enable it to alter the `basic structure' of the Constitution.

\section{Salient Features of the Constitution}

The salient features of the Constitution, as it stands today, are as follows

\subsection{Lengthiest Written Constitution}

Constitution are classified into written, like the American Constitution, or unwritten, like the British Constitution. The Constitution of India is the lengthiest of all the written constitutions of the world. It is very comprehensive, elaborate and detailed document.

Originally (1949), the Constitution contained a Preamble, 395 Articles (divided into 22 Parts) and 8 Schedules. Presently (2016), it consists of a Preamble, about 465 Articles (divided into 25 parts) and 12 Schedules\endnote{For details on Parts, important Articles and Schedules, see Tables 3.1, 3.2 and 3.3 at the end of this chapter.}. The various amendments carried out since 1951 have deleted about 20 Articles and one Part (VII) and added about 90 Articles, four Parts (IVA, IXA, IXB and XIVA) and four Schedules (9, 10, 11 and 12). No other Constitution in the world has so many Articles and Schedules\endnote{The American Constitution originally consisted of only 7 Articles, the Australian 128, the Chinese 138, and the Canadian 147.}.

Four factors have contributed to the elephantine size of our Constitution. They are

\renewcommand{\labelenumi}{\textbf{(\alph{enumi})}}
\begin{enumerate}
  \item Geographical factors, that is, the vastness of the country and its diversity.
  \item Historical factors, e.g., the influence of the Government of India Act of 1935\index{Acts!Government of India Act of 1935}, which was bulky.
  \item Single Constitution for both the Center and the states except Jammu and Kashmir\endnote{The State of Jammu and Kashmir has its own Constitution and thus, enjoys a special status by virtue of Article 370 of the Constitution of India.}.
  \item Dominance of legal luminaries in the Constituent Assembly.
\end{enumerate}

The Constitution contains not only the fundamental principles of governance but also detailed administrative provision. Further, those matters which in other modern democratic countries have been left to the ordinary legislation or established political conventions have also been included in the constitution document itself in India.

\subsection{Drawn from Various Sources}

The Constitution of India has borrowed most of its provisions from the constitutions of various other countries as well as from the Government of India Act\endnote{About 250 provisions of the 1935 Act have been included in the Constitution.} of 1935. \gls{person:DR-B-R-AMBEDKAR} proudly acclaimed that the Constitution of India has been framed after `ransacking all the known Constitution of the World\endnote{Constituent Assembly Debates, Volume VII, P. 35–38.}'.

The structural part of the Constitution is, to a large extent, derived from the Government of India Act of 1935\index{Acts!Government of India Act of 1935}. The philosophical part of the Constitution (the Fundamental Rights\index{default!Fundamental Rights} and the Directive Principles of State Policy) derive their inspiration from the American and Irish Constitution respectively. The political part of the Constitution (the principle of Cabinet Government and the relation between the executive and the legislature) have been largely drawn from the British Constitution\endnote{P M Bakshi, The Constitution of India, Universal, Fifth Edition, 2002, P.4.}.

The other provision of the Constitution have been drawn from the constitutions of Canada, Australia, Germany, USSR (now Russia), France, South Africa, Japan, and so on\endnote{See Table 3.4 at the end of this chapter.}.

The most profound influence and material source of the Constitution is the Government of India Act, 1935. The Federal Scheme, Judiciary, Governors, emergency powers, the Public Service Commissions and most of the administrative details are drawn from this Act. More than half of the provisions of Constitution are identical to or bear a close resemblance to the Act of 1935\endnote{Brij Kishore Sharma, Introduction to the Constitution of India, Seventh Edition, 2015, PHI Learning Private Limited, P.42.}.

\subsection{Blend of Rigidity and Flexibility}

Constitutions are also classified into rigid and flexible. A rigid Constitution is one that requires a special procedure for its amendment, as for example, the American Constitution. A flexible constitution, on the other hand, is one that can be amended in the same manner as the ordinary laws are made, as for example, the British Constitution.

The Constitution of India is neither rigid nor flexible but a synthesis of both. Article 368 provides for two types of amendments

\renewcommand{\labelenumi}{\textbf{(\alph{enumi})}}
\begin{enumerate}
  \item Some provisions can be amended by a special majority of the Parliament, i.e. a two-third majority of the members of each House present and voting, and a majority (that is, more than 50 percent), of the total membership of each House.
  \item Some other provisions can be amended by a special majority of the Parliament and with the ratification by half of the total states
\end{enumerate}

At the same time, some provisions of the Constitution can be amended by a simple majority of the Parliament in the manner of ordinary legislative process. Notable, these amendments do not come under Article 368.

\subsection{Federal System with Unitary Bias}

The Constitution of India establishes a federal system of government. It contains all the usual features of a federation, viz., two government, division of powers, written Constitution, supremacy of Constitution, rigidity of Constitution, independent judiciary and bicameralism\index{default!bicameralism}.

However, the Indian Constitution also contains a large number of unitary or non-federal features, viz., a strong Center, single Constitution, single citizenship, flexibility of Constitution, integrated judiciary, appointment of state governor by the Centre, all-India services, emergency provisions, and so on.

Moreover, the term `Federation' has nowhere been used in the Constitution. Article 1, on the other hand, describes India as a `Union of States' which implies two things: one, Indian Federation is not the result of an agreement by the states; and two, no state has the right to secede from the federation.

Hence, the Indian Constitution has been variously described a `federal in form but unitary in spirit', `quasi-federl' by \gls{person:K-C-WHEARE}, `bargaining federalism' by \gls{person:MORRIS-JONES}, `co-operative federalism' by \gls{person:GRANVILLE-AUSTIN}, `federation with a centralizing tendency' by \gls{person:IVOR-JENNINGS}, and so on.

\subsection{Parliamentary Form of Government}

The Constitution of India has opted for the British parliamentary System of Government rather than American Presidential System of Government. The parliamentary system is based on the principle of cooperation and co-ordination between the legislative and executive organs while the presidential system is based on the doctrine of separation powers between the two organs.

The parliamentary system is also known as the `Westminster'\endnote{Westminster is a place in London where the British Parliament is located. It is often used as a symbol/synonym of the British Parliament.} model of government, responsible government and cabinet government. The Constitution establishes the parliamentary system not only at the Centre but also in the states. The features of parliamentary government in India are

\renewcommand{\labelenumi}{\textbf{(\alph{enumi})}}
\begin{enumerate}
  \item Presence of nominal and real executives,
  \item Majority party rule,
  \item Collective responsibility of the executive to the legislature,
  \item Membership of the minister in the legislature,
  \item Leadership of the prime minister or the chief minister,
  \item Dissolution of the lower House (Lok Sabha or Assembly).
\end{enumerate}

Even though the Indian Parliamentary System is largely based on the British pattern, there are some fundamental difference between the two. For example, the Indian Parliament is not a sovereign body like the British Parliament. Further, the Indian State has an elected head (republic) while the British State has hereditary head (monarchy).

In a parliament system whether in India or Britain, the role of the Prime Minister has become so significant and crucial that the political scientists like to call its a `Prime Ministerial Government'.

\subsection{Synthesis of Parliamentary Sovereignty and Judicial Supremacy}

The doctrine of sovereignty of Parliament is associated with the British Parliament while the principle of judicial supremacy with that of the American Supreme Court.

Just as the Indian parliamentary system differs from the British system, the scope of judicial review power of the Supreme Court in India is narrower than that of what exists in US. This is because the American Constitution provides for `due process of law' against that of `procedure established by law' contained in the Indian Constitution (Article 21).

Therefor, the framers of the Indian Constitution have preferred a proper synthesis between the British principle of parliamentary sovereignty and the American principle of judicial supremacy. The Supreme Court, on the one hand, can declare the parliamentary laws as unconstitutional through its power of judicial review. The Parliament, on the other hand, can amend the major portion of the Constitution through its constituent power.

\subsection{Integrated and Independent Judiciary}

The Indian Constitution established a judicial system that is integrated as well a independent.

The Supreme Court stands at the top of the integrated judicial system in the country. Below it, there are high courts at the state level. Under a high court, there is a hierarchy of subordinate courts, that is, district courts and other lower courts. This single system of courts enforces both the central laws as well as the state laws, unlike in USA, where the federal laws are enforced by the federal judiciary and the state laws are enforced by the state judiciary.

The Supreme Court is a federal court, the highest court of appeal, the guarantor of the fundamental rights of the citizens and the guardian of the Constitution. Hence, the Constitution has made various provisions to ensure its independence — security of tenure of the judges, fixed service conditions for the judges, all the expenses of the Supreme Court charged on the Consolidated Fund of India, prohibition on the discussion on the conduct of judges in the legislatures, ban on practice after retirement, power to punish for its contempt vested in the Supreme Court, separation of the judiciary from the executive, and so on.

\subsection{Fundamental Rights\index{default!Fundamental Rights}}

Part III of the Indian Constitution guarantees six\endnote{Originally, the Constitution provided for seven Fundamental Rights\index{default!Fundamental Rights}. However, the Rightto Property (Article 31) was deleted fromthe list of Fundamental Rights\index{default!Fundamental Rights} by the 44th Amendment Act of 1978. It is made a legal right under Article 300-A in Part XII of the constitution.} fundamental rights to all the citizens

\renewcommand{\labelenumi}{\textbf{(\alph{enumi})}}
\begin{enumerate}
  \item Right to Equality (Article 14-18),
  \item Right to Freedom (Article 19-22),
  \item Right against Exploitation (Article 23-24),
  \item Right to Freedom of Religion (Article 25-28),
  \item Cultural and Educational Right (Article 29-30), and
  \item Right to Constitutional Remedies (Article 32)
\end{enumerate}

The Fundamental Rights\index{default!Fundamental Rights} are meant for promoting the idea of political democracy. They operate as limitations on the tyranny of the executive and arbitrary laws of the legislature. They are justiciable in nature, that is they are enforceable by the courts for their violation. The aggrieved person can directly go to the Supreme Court which can issue the writs of {\colord habeas corpus}, {\colord mandamus}, prohibition, {\colord certiorari} and {\colord quo warranto} for the restoration of his rights.

However, the Fundamental Rights\index{default!Fundamental Rights} are not absolute the subject to reasonable restrictions. Further, they are not sacrosanct and can be curtailed or repealed by the Parliament through a constitutional amendment act. They can also be suspended during the operation of National Emergency except the rights guaranteed by Article 20 and 21.

\subsection{Directive Principles of State Policy}

According to \gls{person:DR-B-R-AMBEDKAR}, the Directive Principles of State Policy is a `novel feature' of the Indian Constitution. They are enumerated in Part IV of the Constitution. They can be classified into three broad categories — socialistic, Gandhian and liberal-intellectual.

The directive principles are meant for promoting the idea of social and economic democracy. They seek to establish a `welfare state' in India. However, unlike the Fundamental Rights\index{default!Fundamental Rights}, the directives are non-justifiable in nature, that is, they are not enforceable by the courts for their violation. Yet, the Constitution itself declares that `these principles are fundamental in the governance of the country and it shall be the duty of the state to apply these principles in making laws'. Hence, they impose a moral obligation on the state authorities for their application. But, the real force (sanction) behind them is political, that is, public opinion.

In the {\colord \textit{Minerva Mills}} case\endnote{Minerva Mills v.Union of India, (1980).} (1980), the Supreme Court held that `the Indian Constitution is founded on the bedrock of the balance between the Fundamental Rights\index{default!Fundamental Rights} and the Directive Principles'.

\subsection{Fundamental Duties}

The original constitution did not provide for the fundamental duties of the citizens. These were added during the operation of internal emergency(1975-77) by the 42nd Constitutional Amendment Act of 1976, on the recommendation of the \gls{committee:SWARAN-SINGH-COMMITTEE}. The 86the Constitutional Amendment Act of 2002, added one more fundamental duty.

The Part IV-A of the Constitution (which consists of only one Article-51 A) specifies the eleven Fundamental Duties,

\renewcommand{\labelenumi}{\textbf{(\alph{enumi})}}
\begin{enumerate}
  \item to respect the Constitution, national flag and national anthem;
  \item to protect the sovereignty, unity and integrity of the country;
  \item to promote the spirit of common brotherhood amongst all the people,
  \item to preserve the rich heritage of our composite culture and so on.
\end{enumerate}

The fundamental duties serve as reminder to citizens that while enjoying their rights, they have also to be quite conscious of duties they owe to their country, their society and to their fellow-citizens. However, like the Directive Principles, the duties are also non-justiciable in nature.


\subsection{A Secular State}

The Constitution of India stands for a secular state. Hence, it does not uphold any particular religion as the official religion of the Indian State. The following provisions of the Constitution reveal the secular character of the Indian State

\renewcommand{\labelenumi}{\textbf{(\alph{enumi})}}
\begin{enumerate}
  \item The term `secular' was added to the Preamble of the Indian Constitution by the 42nd Constitutional Amendment Act of 1976.
  \item The Preamble secures to all citizens of Indian liberty of belief, faith and worship.
  \item The State shall not deny to any person equality before the law or equal protection of the laws (Article 14).
  \item The State shall not discriminate against any citizen on the ground of religion (Article 15).
  \item Equality of opportunity for all citizens in matters of public employment (Article 16).
  \item All persons are equally entitled to freedom of conscience and the right to freely profess, practice and propagate any religion (Article 26).
  \item Every religious denomination or any of its section shall have the right to manage its religious affairs (Article 26).
  \item No person shall be compelled to pay any taxes for the promotion of particular religion (Article 27).
  \item No religious instruction shall be provided in any educational institution maintained by the state (Article 28).
  \item Any section of the citizens shall have the right to conserve its distinct language, script or culture (Article 29).
  \item All minorities shall have the right to establish and administer educational institutions of their choice (Article 30).
  \item The State shall endeavor to secure for all the citizens a Uniform Civil Code (Article 44).
\end{enumerate}

The Western concept of secularism connotes a complete separation between the religion (the church) and the state (the politics). This negative concept of secularism is inapplicable in the Indian situation where the society if multi-religious. Hence, the Indian Constitution embodies the positive concept of secularism, i.e., giving equal respect to all religions or protecting all religions equally.

Moreover, the Constitution has also abolished the old system of communal representation\index{default!communal representation}\endnote{The 1909, 1919, and 1935 Acts provided for communal representation\index{default!communal representation}.}, that is, reservation of seats in the legislature on the basis of religion. However, it provides for the temporary reservation of seats for the scheduled castes and scheduled tribes to ensure adequate representation to them.

\subsection{Universal Adult Franchise\index{default!Adult Franchise}}

The Indian Constitution adopts universal adult franchise\index{default!universal adult franchise} as a basis of elections to the Lok Shabha and the state legislative assemblies. Every citizen who is not less then 18 years of age has a right to vote without any discrimination of caste, race, religion, sex, literacy, wealth, and so on. The voting age was reduced to 18 years from 21 years in 1989 by the 61st Constitutional Amendment Act of 1988.

The Introduction of universal adult franchise\index{default!universal adult franchise} by the Constitution-makers was bold experiment and highly remarkable in view of the vast size of the country, its huge population, high poverty, social inequality and overwhelming illiteracy.\endnote{Even in the western countries, the right to vote was extended only gradually. For example, USA gave franchise to women in 1920, Britain in 1928, USSR (now Russia) in 1936, France in 1945, Italy in 1948 and Switzerland in 1971.}

Universal adult franchise makes democracy broad-based, enhances the self-respect and prestige of the common people, upholds the principle of equality, enables minorities to protect their interests and opens up new hopes and vistas for weaker sections.

\subsection{Single Citizenship}

Through the Indian Constitution is federal and envisages a dual polity (Centre and states), it provides for only a single citizenship, that is, the Indian citizenship.

The countries like USA, on the other hand, each person is not only a citizen of USA but also a citizen of the particular state to which he belongs. Thus, the owes allegiance to both and enjoys dual sets of rights — one conferred by the National government and another by the state government.

In India, all citizens irrespective of the state in which they are born or reside enjoy the same political and civil rights of citizenship all over the country and no discrimination is made between them exception in few cases like tribal areas, Jammu and Kashmir, and so on.

Despite the constitutional provision for a single citizenship and uniform rights for all the people, India has been witnessing the communal riots, class conflicts, caste wars, linguistic clashes and ethnic disputes. This means that the cherished goal of the Constitution-makers to build and united and integrated Indian nation has not been fully realized.

\subsection{Independent Bodies}

The Indian Constitution not only provides for the legislative, executive and judicial organs of the government (Central and state) but also established certain independent bodies. They are envisaged by the Constitution as the bulwarks of the democratic system of Government in India. These are

\renewcommand{\labelenumi}{\textbf{(\alph{enumi})}}
\begin{enumerate}
  \item Election Commission to ensure free and fair elections to the Parliament, the state legislatures, the office of President of India and the office of Vice-president of India.
  \item Comptroller and Auditor-General of India to adult the accounts of the Central and state governments. He acts as the guardian of public purse and comments on the legality and propriety of government expenditure.
  \item Union Public Service Commission to conduct examinations for recruitment to all-India services\endnote{At present, there are three All-India services, namely Indian Administrative Service (IAS), Indian Police Service (IPS) and Indian Forest Service (IFS). In 1947, Indian Civil Service (ICS) was replaced by IAS and the Indian Police (IP) was replaced by IPS and were recognised by the Constitution as All-India Services. In 1963, IFS was created and it came into existence in 1966.} and higher Central services and to advise the President on disciplinary matters.
  \item State Public Service Commission in every state to conduct examinations for recruitment to state services and to advice the governor on disciplinary matters.
\end{enumerate}

The Constitution ensures the independence of these bodies through various provision like security of tenure, fixed service conditions, expenses being charges on the Consolidated Funds of India, and so on.

\subsection{Emergency Provisions}

The Indian Constitution contains elaborate emergency provisions to enable the President to meet any extraordinary situation effectively. The rationality behind the incorporation of these provision is to safeguard the sovereignty, unity, integrity and security of the country, the democratic political system and the Constitution.

The Constitution envisages three types of emergencies, namely

\renewcommand{\labelenumi}{\textbf{(\alph{enumi})}}
\begin{enumerate}
  \item National emergency on the ground of war or externam aggression or armed rebellion\endnote{The 44th Amendment Act (1978) has replaced the original term ‘internal disturbance’ by the new term ‘armed rebellion’.} (Article 352)
  \item State emergency (President's Rule) on the ground of failure of Constitutional machinery in the states (Article 356) or failure to comply with directions of the Centre (Article 365); and
  \item Financial emergency on the ground of threat to the financial stability or credit of India (Article 360).
\end{enumerate}

During an emergency, the Central Government becomes all-powerful and the states go into the total control of the centre. It converts the federal structure into a unitary one without a formal amendment of the Constitution. This kind of transformation of the political system from federal (during normal times) to unitary (during emergency) is unique feature of the Indian Constitution.

\subsection{Three-tier Government}

Originally, the Indian Constitution, like any other federal constitution, provided for a dual polity and contained provision with regard to the organization and powers of the Centre and the states. Later, the 73rd and 74the Constitutional Amendment Act of 1992, has added a third-tier of government (i.e., local) which is not found in any other Constitution of the world.

The 73rd Amendments Act of 1992 gave constitutional recognition to the Panchayat (rural local governments) by adding a new Part IX\endnote{Part IX of the Constitution provides for a three-tier system of panchayati raj in every state, that is, panchayats at the village, intermediate and district levels.} and a new Schedules 11 to the Constitution. Similarly, the 74the Amendment Act of 1992, gave constitutional recognition to the municipalities (urban local government) by adding a new Part IX-A\endnote{Part IX-A of the Constitution provides for three types of municipalities in every state, that is, nagar panchayat for a transitional area, municipal council for a smaller urban area and municipal corporation for a larger urban area.} and a new Schedule 12 to the Constitution.

\subsection{Co-operative Societies}

The $97^{\text{th}}$ Constitutional Amendment Act of 2011, gave a constitutional status and protection to co-operative societies. In this context, it made the following three changes in the Constitution

\begin{enumerate}
  \item It made the right to form co-operative societies a fundamental right (Article 19).
  \item It include a new Directive Principle of State Policy on promotion of co-operative societies (Article 43-B).
  \item It added a new Part IX-B in the Constitution which is entitled as ``The Co-operative societies'' (Article 243-ZH to 243-ZT).
\end{enumerate}

The new Part IX-B contains various provision to ensure that the co-operative societies in the country function in a democratic, professional, autonomous and economically sound manner. It empowers the Parliament in respect of multi-state cooperative societies and the state legislature in respect of other co-operative societies to make the appropriate law.

\section{Criticism of the Constitution}

The Constitution of India as framed and adopted by the Constituent Assembly of India has been criticized on the following grounds

\paragraph{A Borrowed Constitution}

The critics opined that the Indian Constitution contains nothing new and original. They described it as a `borrowed constitution' or a `bag of borrowings' or a `hotch-potch constitution' or a `patchwork' of several documents of the world constitutions. However, this criticism is unfair and illogical. This is because, the framers of the constitution made necessary modification in the features borrowed from other constitutions for their suitability to the Indian conditions, at the same time avoiding their faults.

While answering the above criticism in the Constituent Assembly, \gls{person:DR-B-R-AMBEDKAR}, the Chairman of the \gls{committee:DRAFTING-COMMITTEE}, said: ``One likes to ask whether there can be antthin new in Constitution framed at this hour in the history of the world. More than hundred years have rolled over when the first written constitution was drafted. It has been followed by many countries reducing their constitution to writing. What the scope of a constitution should be has long been settled. Similarly, what are the fundamentals of a constitution are recognized all over the world. Given these facts, all constitutions in thir main provision must look similar. The only new things, if there can be any, in a constitution framed so late in the day are the variations made to remove the faults and to accommodate it to the needs of country. The charge of providing a blind copy of the constitutions of other countries is based, I am sure, on an inadequate study of the Constitution''\endnote{Though the last entry is numbered 284, the actual total number is 282. This is because, three entries (87,92 and 130) have been deleted and one entry is numbered as 257-A.}.

\paragraph{A Carbon Copy of the 1935 Act}

The critics said that the framers of the constitution have included a large number of the provision of the Government of India Act of 1935\index{Acts!Government of India Act of 1935} into the Constitution of India. Hence, they called the constitutions a ``Carbon Copy of the 1935 Act'' or an ``amended version of the 1935 Act''. For example, \gls{person:N-SHRINIVASAN} observed that the Indian Constitution is ``both in language and substance a close copy of the Act of 1935''. Similarly, Sir \gls{person:IVOR-JENNINGS}, a British Constitutionalist, said that ``the constitution derives directly from the Government of India Act of 1935\index{Acts!Government of India Act of 1935} from which, in fact, many of its provisions are copied almost textually''.

Further, \gls{person:P-R-DESHMUKH}, a member of the Constituent Assembly, commented that ``the constitution is essentially the Government of India Act of 1935\index{Acts!Government of India Act of 1935} with only adult franchise added''.

The same \gls{person:DR-B-R-AMBEDKAR} answered the above criticism in the Constituent Assembly in the following way: ``As to the accusation that the Draft Constitution has reproduced a good part of the provision of the Government of India Act, 1935, I make no apologies. There is nothing to be ashamed of in borrowing. It involves no plagiarism. Nobody hold any patent rights in the fundamental ideas of a Constitution. What I am sorry about is that the provision taken from the Government of India Act, 1935, relate mostly to the details of administration''\endnote{Constituent Assembly Debates, Volume VII, pp.35-38.}.

\paragraph{Un-Indian or Anti-India}

According to the critics, the Indian Constitution is `un-Indian' or `anti-Indian' because it does not reflect the political tradition and the spirit of India. They said that the foreign nature of the Constitution makes it unsuitable to the Indian situation or unworkable in India. In this context, \gls{person:K-HANUMANTHAIYA}, a member of the Constituent Assembly, commented: ``We wanted the music of Veena or Sitar, but here we have the music of an English band. That was because out constitution-makers were educated that way''\endnote{Ibid.}. Similarly, \gls{person:K-HANUMANTHAIYA}, another member of the Constituent Assembly, criticized the constitution as ``slavish imitation of the west, much more - a slavish surrender to the west''\endnote{Constituent Assembly Debates, Volume XI, P.616.}. Further, \gls{person:LAXMINARAYAN-SAHU}, also a member of Constituent Assembly, observed: ``The ideals on which this draft constitution is framed have no manifest relation to the fundamental spirit of India. This constitution would not prove suitable and would break down soon after being brought into operation''\endnote{Constituent Assembly Debates, Volume VII, P.242.}.

\paragraph{An Un-Gandhian Constitution}

According to the critics, the Indian Constitution is Un-Gandhian because it does not contain the philosophy and ideals of \gls{person:MAHATMA-GANDHI}, the father of the Indian Nation. They opined that the Constitution should have been raised and build upon village panchayats. In this context, the same member of the Constituent Assembly, \gls{person:K-HANUMANTHAIYA}, said: ``That is exactly the kind of Constitution \gls{person:MAHATMA-GANDHI} did not want and did not envisage''.\endnote{Constituent Assembly Debates, Volume XI, P.613.} \gls{person:T-PRAKASAM}, another member of the Constituent Assembly, attributed this lapse to \gls{person:DR-B-R-AMBEDKAR}'s non-participation in the Gandhian movement and the antagonism towards Gandhian ideas\endnote{Constituent Assembly Debates, Volume XI, P.617.}.

\paragraph{Elephantine Size}

The critics stated that the Indian Constitution is too bulky and too detailed and contains some unnecessary elements. Sir \gls{person:IVOR-JENNINGS}, a British Constitutionalist, observed that the provisions borrowed were not always well-selected and that the constitution, generally speaking, was too long and complicated.\endnote{Constituent Assembly Debates, Volume VII, P.387.}

In this context, \gls{person:H-V-KAMATH}, a member of the Constituent Assembly, commented: ``The emblem and the crest we have selected for our assembly is an elephant. It is perhaps in consonance with that our constitution too is the bulkiest in the world has produced''.\endnote{\gls{person:IVOR-JENNINGS}, Some Characteristics of the Indian Constitution, Oxford University Press, Madras, 1953, PP.9-16.} He also said: ``I am sure, the House does not agree that we should make the Constitution an elephantine one''.\endnote{Constituent Assembly Debates, Volume VII, P.1042.}

\paragraph{Paradise of the Lawyers}

According to the critics, the Indian Constitution is too legalistic and very complicated. They opined that the legal language and phraseology adopted in the constitution makes it a complex document. The same Sir \gls{person:IVOR-JENNINGS} called it a ``lawyer's paradise''.

In this context, \gls{person:H-K-MAHESWARI}, a member of the Constituent Assembly observed: ``The draft tends to make people more litigious, more inclined to go to courts, less truthful and less likely to follow the methods of truth and non-violence. If I may so, the Draft is really a lawyer's paradise. It opens up vast avenues of litigation and will give our able and ingenious lawyers plenty of work to do''.\endnote{Constituent Assembly Debates, Volume VIII, P.127.}

Similarly, \gls{person:P-R-DESHMUKH}, another member of the Constituent Assembly, said: ``I should, however, like to say that the draft of the articles that have been brought before the House by \gls{person:DR-B-R-AMBEDKAR} seems to my mind to be far too ponderous like the ponderous tomes of a law manual. A document dealing with a constitution hardly uses so much of padding and so much of verbiage. Perhaps it is difficult for them to compose a document which should be, to my mind, not a law but a socio-political document, a vibration, pulsating and life giving document. But to our misfortune, that was not to be, and we have been burdened with so much of words, words and words which could have been very easily eliminated''.\endnote{Constituent Assembly Debates, Volume VII, P.293.}

\clearpage

\onecolumn

\begin{longtable}[c]{@{}|p{1cm}|p{7cm}|p{4cm}|@{}}
  \caption{The Constitution of India at a Glance}
  \label{tab:TheConstitutionofIndiaataGlance}\\
  \toprule 
	Parts & Subject Matter & Articles Covered \\
  \bottomrule
  \endfirsthead
  %
  \multicolumn{3}{c}%
  {{\bfseries Table \thetable\ continued from previous page}} \\
  \toprule 
	Parts & Subject Matter & Articles  Covered \\ 
	\bottomrule
  \endhead
  %
  I & The Union and its territory & 1 to 4 \\\midrule
  II & Citizenship & 5 to 11 \\\midrule
  III & Fundamental Rights\index{default!Fundamental Rights} & 12 to 35 \\\midrule
  IV & Directive Principles of State Policy & 36 to 51 \\\midrule
  IV-A & Fundamental Duties & 51-A \\\midrule
  V & The Union Government & 52 to 151 \\
  & Chapter I – The Executive & 52 to 78 \\
  & Chapter II – Parliament & 79 to 122 \\
  & Chapter III – Legislative Powers of President & 123 \\
  & Chapter IV – The Union Judiciary & 124 to 147 \\
  & Chapter V – Comptroller and Auditor-General of Ind & 148 to 151 \\\midrule
  VI & The State Governments & 152 to 237 \\
  & Chapter I – General & 152 \\
  & Chapter II – The Executive & 153 to 167 \\
  & Chapter III – The State Legislature & 168 to 212 \\
  & Chapter IV – Legislative Powers of Governor & 213 \\
  & Chapter V – The High Courts & 214 to 232 \\
  & Chapter VI – Subordinate Courts & 233 to 237 \\\midrule
  VII & The States in Part B of the First Schedule (deleted) & 238 (deleted) \\\midrule
  VIII & The Union Territories & 239 to 242 \\\midrule
  IX & The Panchayat & 243 to 243-0 \\\midrule
  IX-A & The Municipalities & 243-P to 243-ZG \\\midrule
  IX-B & The Co-operative Societies & 243-ZH to 243-ZT \\\midrule
  X & The Scheduled and Tribal Are & 244 to 244-A \\\midrule
  XI & Relations between the Union and the States & 245 to 263 \\
  & Chapter I – Legislative Relations & 245 to 255 \\
  & Chapter II – Administrative Relations & 256 to 263 \\\midrule
  XII & Finance, Property, Contracts and Suits & 264 to 300-A \\*\midrule
  & Chapter I – Finance & 264 to 291 \\
  & Chapter II – Borrowing & 292 to 293 \\
  & Chapter III – Property, Contracts, Rights, &  \\
  & Liabilities, Obligations and Suits & 294 to 300 \\
  & Chapter IV – Right to Proper & 300-A \\ \midrule
  XIII & Trade, Commerce and Intercourse within the Territory of India & 301 to 307 \\\midrule
  XIV & Services under the Union and the Stat & 308 to 323 \\
  & Chapter I – Services & 308 to 314 \\
  & Chapter II – Public Service Commissions & 315 to 323 \\\midrule
  XIV-A & Tribunals & 323-A to 323-B \\\midrule
  XV & Elections & 324 to 329-A \\\midrule
  XVI & Special Provisions relating to Certain Class & 330 to 342 \\\midrule
  XVII & Official Languages & 343 to 351 \\
  & Chapter I – Language of the Union & 343 to 344 \\
  & Chapter II – Regional Languages & 345 to 347 \\
  & Chapter III — Language of the Supreme Court, High Courts, and so & 348 to 349 \\
  & Chapter IV — Special Directives & 350 to 351 \\\midrule
  XVIII & Emergency Provisions & 352 to 360 \\\midrule
  XIX & Miscellaneous & 361 to 367 \\\midrule
  XX & Amendment of the Constitution & 368 \\\midrule
  XXI & Temporary, Transitional and Special Provisions & 369 to 392 \\\midrule
  XXII & Short title, Commencement, Authoritative Text in Hindi and Repeals & 393 to 395\\\bottomrule
\end{longtable}

\clearpage
\begin{longtable}[c]{@{}|p{2cm}|p{10cm}|@{}}
  \caption{Important Articles of the Constitution at a Glance}
  \label{tab:articlesnumberinfo}\\
  \toprule
  Articles & Deals with \\* \midrule
  \endfirsthead
  %
  \multicolumn{2}{c}%
  {{\bfseries Table \thetable\ continued from previous page}} \\
  \toprule
  Articles & Deals with \\* \midrule
  \endhead
  %
  1 & Name and territory of the Union \\ \midrule
  3 & Formation of new states and alteration of areas, boundaries or names of existing states \\ \midrule
  13 & Laws inconsistent with or in derogation of the fundamental rights \\ \midrule
  14 & Equality before law \\ \midrule
  16 & Equality of opportunity in matters of public employment \\ \midrule
  17 & Abolition of untouchability \\ \midrule
  19 & Protection of certain rights regarding freedom of speech, etc. \\ \midrule
  21 & Protection of life and personal liberty \\ \midrule
  21 & A Right to elementary education \\ \midrule
  25 & Freedom of conscience and free profession, practice and propagation of religion \\ \midrule
  30 & Right of minorities to establish and administer educational institutions \\ \midrule
  31—C & Saving of laws giving effect to certain directive principles \\ \midrule
  32 & Remedies for enforcement of fundamental rights including writs \\ \midrule
  38 & State to secure a social order for the promotion of welfare of the people \\ \midrule
  40 & Organisation of village panchayats \\ \midrule
  44 & Uniform civil code for the citizens \\ \midrule
  45 & Provision for early childhood care and education to children below the age of 6 years. \\ \midrule
  46 & Promotion of educational and economic interests of scheduled castes, scheduled tribes and other weaker sections \\ \midrule
  50 & Separation of judiciary from executive \\ \midrule
  51 & Promotion of international peace and security \\ \midrule
  51 & A Fundamental duties \\ \midrule
  72 & Power of president to grant pardons, etc., and to suspend, remit or commute sentences in certain cases \\ \midrule
  74 & Council of ministers to aid and advise the president \\
  78 & Duties of prime minister as respects the furnishing of information to the president, etc. \\ \midrule
  110 & Definition of Money Bills \\ \midrule
  112 & Annual financial statement (Budget) \\ \midrule
  123 & Power of president to promulgate ordinances during recess of Parliament \\ \midrule
  143 & Power of president to consult Supreme Court \\ \midrule
  155 & Appointment of governor \\ \midrule
  161 & Power of governor to grant pardons, etc., and to suspend, remit or commute sentences in certain cases \\ \midrule
  163 & Council of ministers to aid and advise the governor \\ \midrule
  167 & Duties of chief minister with regard to the furnishing of information to governor, etc. \\ \midrule
  169 & Abolition or creation of legislative councils in states \\ \midrule
  200 & Assent to bills by governor (including reservation for President) \\ \midrule
  213 & Power of governor to promulgate ordinances during recess of the state legislature \\ \midrule
  226 & Power of high courts to issue certain writs \\ \midrule
  239 & AA Special provisions with respect to Delhi \\ \midrule
  249 & Power of Parliament to legislate with respect  to a matter in the State List in the national interest \\ \midrule
  262 & Adjudication of disputes relating to waters of inter-state rivers or river valleys \\ \midrule
  263 & Provisions with respect to an inter-state council \\ \midrule
  265 & Taxes not to be imposed save by authority of law \\ \midrule
  275 & Grants from the Union to certain states \\ \midrule
  280 & Finance Commission \\ \midrule
  300 & Suits and proceedings \\ \midrule
  300 & A Persons not to be deprived of property save by authority of law (Right to property) \\ \midrule
  311 & Dismissal, removal or reduction in rank of persons employed in civil capacities under the Union or a state. \\ \midrule
  312 & All-India Services \\ \midrule
  315 & Public service commissions for the Union and for the states \\ \midrule
  320 & Functions of Public service commissions \\ \midrule
  323-A & Administrative tribunals \\ \midrule
  324 & Superintendence, direction and control of elections to be vested in an Election Commission \\ \midrule
  330 & Reservation of seats for scheduled castes and scheduled tribes in the House of the People \\ \midrule
  335 & Claims of scheduled castes and scheduled tribes to services and posts \\ \midrule
  352 & Proclamation of Emergency (National Emergency) \\ \midrule
  356 & Provisions in case of failure of constitutional machineryin states (President’s Rule) \\ \midrule
  360 & Provisions as to financial emergency. \\ \midrule
  365 & Effect of failure to comply with, or to give effect to, directions given by the Union (President’s Rule) \\ \midrule
  368 & Power of Parliament to amend the Constitution and procedure therefor \\ \midrule
  370 & Temporary provisions with respect to the state of  Jammu and Kashmir\\ \midrule
\end{longtable}

\theendnotes
\cleardoublepage\endnote{Constituent Assembly Debates, Volume IX, P.613.}
% From File: B:/Writing/Books/Indian_Polity_V2/TeX_files/01_04.tex
%

\twocolumn

\chapter{Preamble of the Constitution}

The American Constitution was the first to begin with a Preamble. Many countries, including India, followed this practice. The term `preamble' refers to the introduction or preface to the Constitution. It contains the summary or essence of the Constitution. \gls{person:N-A-PALKIVALA}, an eminent jurist and constitutional expert, called the Preamble as the `identity card of the Constitution'.

The Preamble to the Indian Constitution is based on the `Objectives Resolution\index{default!Objectives Resolution}', draft and moved by \gls{person:JAWAHARLAL-NEHRU}, and adopted by the Constituent Assembly\endnote{Moved by Nehru on December 13, 1946 and adopted by the Constituent Assembly on January 22, 1947.}. It has been amended by the 42nd Constitutional Amendment Act of 1976, which added three new words — socialist, secular and integrity.

\section{Text of the Preamble}

The Preamble in its present form reads,

\textquotedblleft \textbf{We, THE PEOPLE OF INDIA}, having solemnly resolved to constitute\\India into a \textbf{SOVEREIGN SOCIALIST SECULAR DEMOCRATIC REPUBLIC} and to secure to all its citizens;
\begin{list}{}{}
  \item \textbf{JUSTICE}, social, economical and political;
  \item \textbf{LIBERTY} of thought, expression, belief, faith and worship;
  \item \textbf{EQUALITY} of status and of opportunity; and to promote among them all;
  \item \textbf{FRATERNITY} assuring the dignity of the individual and the \textit{unity and integrity} of the Nation;
\end{list}
\textbf{IN OUR CONSTITUENT ASSEMBLY} this twenty-sixth day of 1949-11-00, do \textbf{HEREBY ADOPT, ENACT AND GIVE TO OURSELVES THIS CONSTITUTION}.\textquotedblright

\section{Ingredient of the Preamble}

The Preamble reveals four ingredients or components

\begin{enumerate}
  \item \textbf{Source of authority of the Constitution}: The Preamble states that the Constitution derives its authority from the people of India.
  \item \textbf{Nature of Indian State}: It declares India to be of sovereign, socialist, secular democratic and republican polity.
  \item \textbf{Objectives of the Constitution}: It specifies justice, liberty, equality and fraternity as the objectives.
  \item \textbf{Date of adoption of the Constitution}: It stipulates 1949-11-26, as the date.
\end{enumerate}


\section{Key Words in the Preamble}

Certain key words — Sovereign,, Secular, Democratic, Republic, Justice, Liberty, Equality and Fraternity — are explained as follows,

\subsection{Sovereign}

The word `sovereign' implies that India is neither a dependency nor a dominion of any other nation, but an independent state\endnote{Till the passage of the Indian Independence Act, 1947, India was a dependency (colony) of the British Empire. From August 15, 1947 to January 26, 1950, India’s political status was that of a dominion in the British Commonwealth\index{default!British Commonwealth} of Nations. India ceased to be a British dominion on January 26, 1950, by declaring herself a sovereign republic. However, Pakistan continued to be a British Dominion until 1956.}. There is no authority above it, and it is free to conduct its own affairs (both internal and external).

Though in 1949, India declared the continuation of her full membership of the Commonwealth of Nations and accepted the British Crown as the head of the Commonwealth, this extra-constitutional declaration does not affect India's membership of the United Nations Organization (UNO) also in no way constitutes a limited on her sovereignty\endnote{To dispel the lurking fears of some members of the Constituent Assembly, Pandit Nehru said in 1949 thus: ‘We took pledge long ago to achieve Purna Swaraj. We have achieved it. Does a nation lose its independence by an alliance with another country? Alliance normally means commitments. The free association of the sovereign Commonwealth of Nations does not involve such commitments. Its very strength lies in its flexibility and its complete freedom. It is well-known that it is open to any member-nation to go out of the commonwealth if it so chooses’. He further stated, ‘It is an agreement by free will, to be terminated by free will’.}.

Being a sovereign state, India can either acquire a foreign territory or cede a part of its territory of a foreign state.

\subsection{Socialist}

Even before the term was added by the 42nd Amendment of 1976, the Constitution had a socialist content in the form of certain Directive Principles of State Policy. In other words, what was hitherto implicit in the Constitution has now been made explicit. Moreover, the Congress party itself adopted a resolution\endnote{India became a member of the UNO in 1945.} to establish a `socialistic pattern of society' in its Avadi session as early as in 1955 and took measures accordingly. Notably, the Indian brand of socialism is a `democratic socialism' and not a `communistic socialism' (also known as `state socialism') which involves the nationalisation of all means of production and distribution and the abolition of private property. Democratic socialism, on the other hand, holds faith in a `mixed economy' where both public and private sectors co-exist side by side\endnote{The Resolution said: ‘In order to realise the object of Congress and to further the objectives stated in the Preamble and Directive Principles of State Policy of the Constitution of India, planning should take place with a view to the establishment of a socialistic pattern of society, where the principal means of production are under social ownership or control, production is progressively speeded up and there is equaitable distribution of the national wealth’.}. As the Supreme Court says, `Democratic socialism aims to end poverty, ignorance, disease and inequality of opportunity\endnote{The Prime Minister, Indira Gandhi, said, ‘We have always said that we have our own brand of socialism. We will nationalise the sectors where we feel the necessity. Just nationali-sation is not our type of socialism’.}. Indian socialism is a blend of Marxism and Gandhi-ism, leaning heavily towards Gandhian socialism'. The new economic policy (1991) of liberalization, privatization and globalization has, however, diluted the socialist credentials of the Indian State.


\theendnotes
\cleardoublepage\endnote{G.B. Pant University of Agriculture and Technology v. State of Uttar Pradesh (2000).}
% From File: B:/Writing/Books/Indian_Polity_V2/TeX_files/02_00.tex
%


