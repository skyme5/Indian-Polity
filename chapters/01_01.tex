\twocolumn

\chapter{Historical Background}

The British came to India in 1600 as traders, in the form of East India Company, which had the exclusive right of trading in India under a chapter granted by Queen Elizabeth I. In 1764, the Company, which till now had purely trading functions obtained the `diwani' (i.e., rights over revenue and civil justice) of Bengal, Bihar and Orissa
\endnote{The Mughal Emperor, Shah Alam, granted \quotesingle{Diwani} to the Company after its victory in the Battle of Buxar (1764)}. This started its career as a territorial power. In 1858, in the wake of the \quotesingle{Sepoy mutiny}, the British Crown assumed direct responsibility for the governance of India. This rule continued until India was granted independence on August 15, 1947.

With Independence came the need of a Constitution. As suggested by M.N. Roy (a pioneer of communist movement in India and an advocate of Radical Democrat-ism) in 1934, a Constituent Assembly was formed for this purpose in 1946-00-00 and on 1950-01-26 the Constitution and polity have their roots in the British rule. There are certain events in the British rule that laid down the legal framework for the organization and functioning of government and administration in British India. These events have greatly influenced out contribution and polity. They are explained here in a chronological order:

\section{The Company Rule (1973-1858)}

\subsection{Regulating Act of 1773}
This act is of great constitutional importance as (a) it was the first step taken by the British Government to control and regulate the affairs of the East India Company in India; (b) it recognized, for the first time, the political and administrative functions of the Company; and (c) it laid the foundations of central administration in India.

\paragraph{Features of the Act}

\begin{enumerate}
  \item It designated the Governor of Bengal as the \quotesingle{Governor-General of Bengal} and created an Executive Council of four members to assist him. The first such Governor-General was Lord Warren Hastings.
  \item It made the governors of Bombay and Madras presidencies subordinate to the governor-general of Bengal, unlike earlier, when the three presidencies were independent of one another.
  \item It provided for the establishment of a Supreme Court at Calcutta (1774) comprising one chief justice and three other judges.
  \item It prohibited the servants of the Company from engaging in any private trade or accepting presents or bribes from the \quotesingle{natives}.
  \item It strengthened the control of the British Government over the Company by requiring the Court of Directors (governing body of the Company) to report on its revenue, civil, and military affairs in India.
\end{enumerate}

\subsection{Pitt's India Act of 1784}

In a bid to rectify the defects of the Regulating Act of 1773, the British Parliament passed the Amending Act of 1781, also known as the Act of Settlement. The next important act was the Pitt's India Act\endnote{It was introduced in the British Parliament by the then Prime Minister, William Pitt} of 1784.

\paragraph{Features of the Act}
\begin{enumerate}
  \item It distinguished between the commercial and political functions of the company.
  \item It allowed the Court of Directors to manage the commercial affairs but created a new body called Board of Control to manage the political affairs. Thus, it established a system of double government.
  \item It employed the Board of Control to supervise and direct all operations of civil and military government or revenues of the British possessions in India.
\end{enumerate}

Thus, the act was significant for two reasons: first, the Company's territories in India were for the first time called the \quotesingle{British possessions in India}; and second, the British Government was given the supreme control over Company's affairs and its administration in India.

\subsection{Charter Act of 1833}

This Act was the final step towards the centralization in British India.

\paragraph{Features of the Act}
\begin{enumerate}
  \item It made the Governor-General of Bengal as the Governor-General of India and vested in him all civil and military powers. Thus, the act created, for the first time, a Government of India having authority over the entire territorial area possessed by the British in India. Lord William Bentick was the first governor-general of India.
  \item It deprived the governor of Bombay and Madras of their legislative powers. The Governor-General of India was given exclusive legislative powers for the entire British India. The law made under the previous acts were called as Regulation while laws made under this act were called as Acts.
  \item It ended the activities of the East India Company as a commercial body, which became a purely administrative body. It provided that the company's territories in India were held by it \quotesingle{in trust for His Majesty, His heirs and successors}.
  \item The Charter Act of 1833 attempted to introduce a system of open competition for civil servants, and stated that the Indians should not be debarred from holding any place, office and employment under the Company. However, this provision negated after opposition from the Courts of Directors.
\end{enumerate}

\subsection{Charter Act of 1853}

This was the last of the series of Charter Acts passed by the British Parliament between 1793 and 1853. It was a significant constitutional landmark.
\paragraph{Features of the Act}

\begin{enumerate}
  \item It separated, for the first time, the legislative and executive functions of the Governor-General's council. It provided for the addition of six new members called the legislative councilors to the council. In other words, it established a separate Governor-General's legislative council which came to be known as the Indian (Central) Legislative Council. This legislative wing of the council functioned as mini-Parliament, adopting the same procedures as the British Parliament. Thus, legislative, of the first time, was treated as a special function of the government, requiring special machinery and special process.
  \item It introduced an open competition system of selection and recruitment of civil servants\endnote{At that time, the Civil Services of the company were classified into covenanted civil services (higher civil services) and uncovenanted civil services (lower civil services). The former was created by a law of the Company, while the lated was created otherwise.} was thus thrown open to Indians also. Accordingly, the Macaulay Committee (the Committee on the Indian Civil Services) was appointed in 1854.
  \item It extended the Company's rule and allowed it to retain the possession of Indian territories on trust for the British Crown. But, it did not specify any particular period, unlike the previous Charters. This was a clear indication that the Company' rule could be terminated at any time the Parliament liked.
  \item It introduced, for the first time, local representation in the Indian (Central) Legislative Council. Of the six new legislative members of the governor-general's council, four members were appointed by the local (provincial) governments of Madras, Bombay, Bengal and Agra.
\end{enumerate}

\section{The Crown Rule}

\subsection*{Government of India Act of 1858}

This significant Act was enacted in the wake of the Revolt of 1857 - also known as the First War of Independence of the \quotesingle{Sepoy mutiny}. The act known as the \textbf{Act for the Good Government of India}, abolished the East India Company, and transferred the powers of government, territories and revenues to British Crown.

\paragraph{Features of the Act}
\begin{enumerate}
  \item It provided that India henceforth was to be governed by, and in the name of, Her Majesty. It changed the designation of th Governor-General of India to that of Viceroy of India. He (viceroy) was the direct representative of the British Crown in India. Lord Canning thus became first Viceroy of India.
  \item It ended the system of double government by abolishing the Board of Control and Court of Directors.
  \item It created a new office, Secretary of State for India, vested with complete authority and control over Indian administration. The secretary of state was a member of the British cabinet and was responsible ultimately to the British Parliament.
  \item It established a 15-member Council of India to assist the secretary of state for India. The council was an advisory body. The secretary of state was made the chairman of the council.
  \item It constituted the secretary of state-in-council as a body corporate, capable of suing and being sued in India and in England.
\end{enumerate}

\quotesingle{The Act of 1858 was, however, largely confined to improvement of the administrative machinery by which the Indian Government was to be supervised and controlled in England. It did not alter in any substantial way the system of government that prevailed in India.\endnote{Subhash C. Kashyap, \textit{Out Constitution}, National Book Trust, 3rd Edition, 2001, P. A-10}}


\subsection{Indian Councils Act of 1861, 1892 and 1909}

After the great revolt of 1857, the British Government felt the necessity of seeking the cooperation of the Indians in the administration of their country. In pursaunce of this policy of association, three acts were enacted by the British Parliament in 1864, 1892 and 1909. The Indian Council Act of 1864 is an important landmark in the constitutional and political history of India.

\paragraph{Features of the Act of 1861}
\begin{enumerate}
  \item It made a beginning of representative institutions by association Indians with the law-making process. It thus provided that the viceroy should nominate some Indians as non-official members of his expanded council. In 1862, Lord Canning, the then viceroy, nominated three Indians to his legislative council — the Raja of Benaras, the Maharaja of Patiala and Dinkar RaoSir Dinkar Rao.
  \item It initiated the process of decentralization by restoring the legislative powers to the Bombay and Madras Presidencies. It thus reversed the centralizing tendency that started from the Regulating Act of 1773 and reached its climax under the Charter Act of 1833. This policy of legislative devolution resulted in the grant of almost complete internal autonomy to the provinces in 1937.
  \item It also provided for the establishment of new legislative councils for Bengal, North-Western Frontier Province (NWFP) and Panjab, which were established in 1862, 1866 and 1897 respectively.
  \item It empowered the Viceroy to make rules and orders for the more convenient transaction of business in the council. It also gave a recognition to the  \quotesingle{portfolio} system, introduced by Lord CanningLord Canning in 1859. Under this, a member of the Viceroy's council was made in-charge of one or more departments of the government and was authorized to issue final orders on behalf of the council on matters of his department(s).
  \item It empowered the Viceroy to issue ordinance, without the concurrence of the legislative council, during an emergency. The life of such ordinance was six months.
\end{enumerate}

\paragraph{Features of the Act of 1892}
\begin{enumerate}
  \item It increased the number of additional (non-official) members in the Central and provincial legislative councils, but maintained the official majority in them.
  \item It increased the function of legislative councils and gave them the power of discussing the budget\endnote{The system of Budget was introduced in British India in 1860.} and addressing questions to executive.
  \item It provided for the nomination of some non-official members of the 
  \begin{list}{}{}
    \item[(a)] Central Legislative Council by the viceroy on the recommendation of the provincial legislative councils and the Bengal Chamber of Commerce,
    \item[(b)] that of the Provincial legislative councils by the Governors on the recommendation of the district boards, municipalities, universities, trade associations, zamindars and chambers.
  \end{list}
\end{enumerate}
\quotesingle{The act made a limited and indirect provision for the use of election in filling up some of the non-official seats both in the Central and provincial legislative councils. The word \quotedouble{election} was, however not used in the act. The process was described as nomination made on the recommendation of certain bodies\endnote{V. N. Shukla, The Constitution of India, Esatern Book Company, 10th Edition, 2001, P. A-10.}.}

\paragraph{Features of the Act of 1909}
This Act is also known as Morley-Minto Reforms (Lord Morley was the then Secretary of State for India and Lord Minto was the then Viceroy of India).
\begin{enumerate}
  \item It considerably increased the size of the legislative councils, both Central and Provincial. The number of members in the Central Legislative Council was raised from 16 to 60. The number of members in the provincial legislative councils was not uniform.
  \item It retained official majority in the Central Legislative Council but allowed the provincial legislative councils to habe non-official majority.
  \item It enlarged the deliberative functions of the legislative councils at both the levels. For example, members were allowed to ask supplementary questions, move resolutions on the budget, and so on.
  \item It provided (for the first time) for the association of Indians with the executive Councils of the Viceroy and Governors. Satyendra Prasad Sinha became the first Indian to join the Viceroy's Executive Council. He was appointed as the law member.
  \item It introduced a system of communal representation for Muslims by accepting the concept of \quotesingle{Separate Electorate}. Under this, the Muslim members were to be elected only by Muslim voter. Thus, the Act \quotesingle{legalized communal-ism} and Lord Minto came to be known as the Father of Communal Electorate.
  \item It also provided for the separate representation of presidency corporations, chambers of commerce, universities and zamindars.
\end{enumerate}

\subsection{Government of India Act of 1919}

On August 20, 1917, the British Government declared, for the first time, that its objective was the gradual introduction of responsible government in India\endnote{The declaration this stated: \quotesingle{The policy of His Majesty's Government is that of the increasing association of Indians in every branch of the administration, and the gradual development of self-government institutions, with a view to the progressive realization of responsible government in India as an integral part of the British Empire}.}.

The Government of India Act of 1919 was thus enacted, which came into force in 1921. This Act is also known as Montagu-Chelmsford Reforms (Montagu was the Secretary of State for India and Chelmsford was the Viceroy of India).

\paragraph{Features of the Act}
\begin{enumerate}
  \item It relaxed the central control over the provinces by demarcating and separating the central and provincial subjects. The central and provincial legislatures were authorized to make laws on their respective list of subjects. However, the structure of government continued to be
  centralized and unitary.
  \item It further divided the provincial subjects into two parts - transferred and reserved. The transferred subjects were to be administered by the governor with the aid of ministers responsible to the legislative Council. The reserved subjects, on the other hand, were to be administered by the governor and his executive council without being responsible to the legislative Council. This dual scheme of governance was known as
   \quotesingle{dyarchy} - a term derived from the Greek word di-arche which means double rule. However, this experiment was largely unsuccessful.
  \item It introduced, for the first time, bicameralism and direct elections in the country. Thus, the Indian Legislative Council was replaced by bicameral legislative consisting of an Upper House (Council of State) and a Lower House (Legislative Assembly). The majority of members of both the Houses were chosen by direct election.
  \item It required that the three of the six members of the Viceroy's executive council (other than the commander-in-chief) were to be Indian.
  \item It extended the principle of communal representation by providing separate electorates for Sikhs, Indian Christias, Anglo-Indians and Europeans.
  \item It granted franchise to a limited number of people on the basis of property, tax or education.
  \item It created a new office of the High Commissioner for India in London and transferred to him some of the functions hitherto performed by the Secretary of State for India.
  \item It provided for the establishment of a public service commission. Hence, a Central Public Service Commission was set up in 1926 for recruiting civil servants\endnote{This was done on the recommendation of the Lee Commission on Superior Civil Services in India (1923–24).}.
  \item It separated, for the first time, provincial budgets from the Central budget and authorized the provincial legislatures to enact their budgets.
  \item It provided for the appointment of a statutory commission to inquire into and report on its working after ten years of its coming into force.
\end{enumerate}

\paragraph{Simon Commission}

In November 1927 itself (i.e. 2 years before the schedule), the British Government announced the appointment of a seven member statutory commission under the chairmanship of Sir John Simon to report on the condition of India under its new Constitution. All the members of the commission were British and hence, all the parties boycotted the commission, The commission submitted its report in 1930 and recommended the abolition of dyarchy, extension of responsible government in the provinces, establishment of a federation of British India and princely states, continuation of communal electorate and so on. To consider the proposal of the commission, the British Government convened three round table conferences of the representatives of the British Government, British India and Indian princely state. On the basis of these discussions, a \quotesingle{White Paper on Constitutional Reforms} was prepared and submitted for the consideration of the Joint Select Committee of the British Parliament. The recommendations of this committee were incorporated (with certain changes) in the next Government of India Act of 1935.

\paragraph{Communal Award}
In August 1932, Ramsay MacDonald, the British Prime Minister, announced a scheme of representation of the minorities, which came to be known as the Communal Award. The award not only continued separate electorates for the Muslims, Sikhs, Indian Christians, Anglo-Indians and Europeans but also extended it to the depressed classes (schedules castes). Mahatma Gandhi was distressed over this extension of the principle of communal representation to the depressed classes and undertook fast unto death in Yervada Jail (Poona) to get the award modified. At last, there was an agreement between the leaders of the Congress and the depressed classes. The agreement, known and Poona Pact, retained the Hindu joint electorate and gave reserved seats to the depressed classes.



\subsection{Government of India Act of 1935}

The Act marked a second milestone towards a completely responsible government in India. It was a lengthy and detailed document having 321 Sections and 10 Schedules.

\paragraph{Features of the Act}
\begin{enumerate}
  \item It provided for the establishment of an All-India Federation consisting of provinces and princely states as units. The Act divided the powers between the Centre and units in terms of three lists - Federal List (for Center, with 59 items), Provincial List (for provinces, with 54 items) and the Concurrent List (for both, with 36 items). Residuary powers were given to the Viceroy. However, the federation never came into being as the princely states did not join it.
  \item It abolished dyarchy in the provinces and introduced \quotesingle{provincial autonomy} in its place. The provinces were allowed to act as autonomous units of administration in their defined spheres. Moreover, the Act introduced responsible governments in provinces, that is, the governor was required to act with the advice of ministers responsible to provincial legislature. This came into effect in 1937 and was discontinued in 1939.
  \item It provided for the adoption of dyarchy at the Centre. Consequently, the federal subjects were divided into reserved subjects and transferred subjects. However, this provision of the Act did not come into operation at all.
  \item It introduced bicameralism in six out of eleven provinces. Thus, the legislature of Bengal, Bombay, Madras, Bihar, Assam and the United Provinces were made bicameral consisting of a legislative council (upper house) and a legislative assembly (lower house). However, many restrictions were placed on them.
  \item It further extended the principle of communal representation by providing separate electorates for depressed classes (schedules castes), women and labor (workers).
  \item It abolished the Council of India, established by the Government of India Act of 1858. The secretary of state for India was provided with a team of advisors.
  \item It extended franchise. About 10 percent of the total population got the voting right.
  \item It provided for the establishment of Reserve Bank of India to control the currency and credit of the country.
  \item It provided for the establishment of not only Federal Public Service Commission but also a Provincial Public Service Commission and Joint Public Service Commission for two or more provinces.
  \item It provided for the establishment of a Federal Court, which was set up in 1937.
\end{enumerate}


\subsection{Indian Independence Act of 1947}

On Feburary 20, 1947, the British Prime Minister Clement Atlee declared that the British rule in India would end by June 20, 1948; after which the power would be transferred to responsible Indian hands. This announcement was followed by the agitation by the Muslim League demanding partition of the country. Again on June 3, 1947, the British Government made it clear that any Constitution framed by the Constituent Assembly of India (formed in 1946) cannot apply to those parts of the country which were unwilling to accept it. On the same day (June 3, 1947), Lord Mountbetten, the viceroy of India, put forth the partition plan known as the Mountbatten Plan. The plan was accepted by the Congress and the Muslim League. Immediate effect was given to the plan by enacting the Indian Independence Act\endnote{The Indian Independence Bill was introduced in the British Parliament on July 4, 1947 and received the Royal Assent on July 18, 1947. The act came into force on August 15, 1947.} (1947).

\paragraph{Features of the Act}
\begin{enumerate}
  \item It ended the British rule in India and declared India as an independent and sovereign state from August 15, 1947.
  \item It provided for the partition of India and creation of two independent dominions of India and Pakistan with the right to secede from the British Commonwealth.
  \item It abolished the office of viceroy and provided, for each dominion, a governor-general, who was to be appointed by the British King on the advice of the dominion cabinet. His Majesty's Government in Britain was to have no responsibility with respect to the Government of India or Pakistan.
  \item It empowered the Constituent Assemblies of both the dominions to legislate for their respective territories till the new constitutions were drafted and enforced. No Act of the British Parliament passed after 15, 1947 was to extend to either of the new dominion unless it was extended thereto by a law of the legislature of the dominion.
  \item It abolished the office of the secretary of state for India and transferred his function to the secretary of state for Commonwealth Affairs.
  \item It proclaimed the lapse of British paramountcy over the Indian princely states and treaty relations with tribal areas from August 15, 1947.
  \item It granted freedom to the Indian princely state either to join the Dominion of India or Dominion of Pakistan or to remai independent.
  \item It provided for the Governance of each of the dominions and the provinces by the Government of India Act of 1935, till the new Constitutions were framed. The dominions were however authorized to make modifications in the Act.
  \item It deprived the British Monarch of his right to veto bills or ask for reservation of certain bills for his approval. But, this right was reserved for the Governor-General. The Governor-General would have full power to assent to any bill in the name of His Majesty.
  \item It designated the Governor-General of Indian and the provincial governors as constitutional (nominal) heads of the states. They were made to act on the advice of the respective council of ministers in all matters.
  \item It discontinued the appointment to civil services and reservation of posts by the secretary of state for India. The members of the civil services appointed before August 15, 1947 would continue to enjoy all benefits that they were entitled to till that time.
\end{enumerate}

At the stroke of midnight 14-15 August, 1947, the British rule came to an end and power was transferred to the two new independent Dominions of India and Pakistan\endnote{The boundaries between the two Dominions were determined by a Boundary Commission headed by Radcliff. Pakistan included the provinces of West Punjab, Sind, Baluchistan, East Bengal, North-Western Frontier Province and the district of Sylhet in Assam. The referendum in the North-Western Frontier Province and Sylhet was in favor of Pakistan.}. Lord Mountbetten became the first governor-general of the new Dominion of India. He swore in Jawaharlal Nehru as the first prime minister of independent India. The Constituent Assembly of India formed in 1946 became the Parliament of the Indian Dominion.

\onecolumn

\begin{longtable}[c]{@{}|p{1cm}|p{4cm}|p{7cm}|@{}}
  \caption{Interim Government (1946)}
  \label{tbl:InterimGovernment}\\
  \toprule
  Sl. No. & Members & Portfolios Held \\* \midrule
  \endfirsthead
  %
  \multicolumn{3}{c}%
  {{\bfseries Table \thetable\ continued from previous page}} \\
  \toprule
  Sl. No. & Members & Portfolios Held \\* \midrule
  \endhead
  %
  \bottomrule
  \endfoot
  %
  \endlastfoot
  %
  1. & Jawaharlal Nehru & External Affairs \& Commonwealth \\
     &                           & Relations \\
  2. & Sardar Vallabhbhai Patel & Home, Information \& Broadcasting \\
  3. & Dr. Rajendra Prasad & Food \& Agriculture \\
  4. & Dr. John Mathai & Industries \& Supplies \\
  5. & Jagjivan Ram & Labor \\
  6. & Sardar Baldev Singh & Defense \\
  7. & C.H. Bhabha & Works, Mines \& Power \\
  8. & Liaquat Ali Khan & Finance \\
  9. & Abdur Rab Nishtar & Posts \& Air \\
  10. & Asaf Ali & Railways \& Transport \\
  11. & C. Rajagopalachari & Education \& Arts \\
  12. & I.I. Chundrigar & Commerce \\
  13. & Ghaznafar Ali Khan & Health \\
  14. & Joginder Nath Mandal & Law \\* \bottomrule
\end{longtable}
\textit{\textbf{Note}}: The members of the interim government were members of the Viceroy's Executive Council. The Viceroy continues to be the head of the Council. But, Jawaharlal Nehru was designated as the Vice-President of the Council.

\begin{longtable}[c]{@{}|p{1cm}|p{4cm}|p{7cm}|@{}}
	\caption{First Cabinet of Free India (1947)}
	\label{tbl:firstCabinetOfFreeIndia}\\
	\toprule
	Sl. No. & Members & Portfolios Held \\* \midrule
	\endfirsthead
	%
	\multicolumn{3}{c}%
	{{\bfseries Table \thetable\ continued from previous page}} \\
	\toprule
	Sl. No. & Members & Portfolios Held \\* \midrule
	\endhead
	%
	\bottomrule
	\endfoot
	%
	\endlastfoot
	%
	1. & Jawaharlal Nehru & Prime Minister; External Affairs \& Commonwealth \\
	&                           & Relations; Scientific Research \\
	2. & Sardar Vallabhbhai Patel & Home, Information \& Broadcasting; States \\
	3. & Dr. Rajendra Prasad & Food \& Agriculture \\
	4. & Maulana Abul Kalam Azad & Education \\
	5. & Dr. John Mathai & Railway \& Transport \\
	6. & R.K. Shanmugham & Finance \\
	7. & Dr. B.R. Ambedkar & Law \\
	8. & Jagjivan Ram & Labor \\
	9. & Sardar Baldev Singh & Defense \\
	10. & Raj Kumari Amrit Kaur & Health \\
	11. & C.H. Bhabha & Commerce \\
	12. & Rafi Ahmed Kidwai & Communication \\
	13. & Dr. Shyama Prasad Mukherji & Industries \& Supplies \\
	14. & V.N. Gadgil & Works, Mines \& Power \\* \bottomrule
\end{longtable}

\theendnotes
\cleardoublepage