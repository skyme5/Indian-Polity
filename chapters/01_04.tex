\twocolumn

\chapter{Preamble of the Constitution}

The American Constitution was the first to begin with a Preamble. Many countries, including India, followed this practice. The term \quotesingle{preamble} refers to the introduction or preface to the Constitution. It contains the summary or essence of the Constitution. N A Palkivala, an eminent jurist and constitutional expert, called the Preamble as the \quotesingle{identity card of the Constitution.}

The Preamble to the Indian Constitution is based on the \quotesingle{Objectives Resolution}, draft and moved by Jawaharlal Nehru, and adopted by the Constituent Assembly\endnote{1}. It has been amended by the 42nd Constitutional Amendment Act of 1976, which added three new words — socialist, secular and integrity.

\section{Text of the Preamble}

The Preamble in its present form reads,

\textquotedblleft \textbf{We, THE PEOPLE OF INDIA}, having solemnly resolved to constitute\\India into a \textbf{SOVEREIGN SOCIALIST SECULAR DEMOCRATIC REPUBLIC} and to secure to all its citizens;
\begin{list}{}{}
  \item \textbf{JUSTICE}, social, economical and political;
  \item \textbf{LIBERTY} of thought, expression, belief, faith and worship;
  \item \textbf{EQUALITY} of status and of opportunity; and to promote among them all;
  \item \textbf{FRATERNITY} assuring the dignity of the individual and the \textit{unity and integrity} of the Nation;
\end{list}
\textbf{IN OUR CONSTITUENT ASSEMBLY} this twenty-sixth day of 1949-11-00, do \textbf{HEREBY ADOPT, ENACT AND GIVE TO OURSELVES THIS CONSTITUTION}.\textquotedblright

\section{Ingredient of the Preamble}

The Preamble reveals four ingredients or components

\begin{enumerate}
  \item \textbf{Source of authority of the Constitution}: The Preamble states that the Constitution derives its authority from the people of India.
  \item \textbf{Nature of Indian State}: It declares India to be of sovereign, socialist, secular democratic and republican polity.
  \item \textbf{Objectives of the Constitution}: It specifies justice, liberty, equality and fraternity as the objectives.
  \item \textbf{Date of adoption of the Constitution}: It stipulates 1949-11-26, as the date.
\end{enumerate}


\section{Key Words in the Preamble}

Certain key words — Sovereign,, Secular, Democratic, Republic, Justice, Liberty, Equality and Fraternity — are explained as follows,

\subsection{Sovereign}

The word \quotesingle{sovereign} implies that India is neither a dependency nor a dominion of any other nation, but an independent state\endnote{2}. There is no authority above it, and it is free to conduct its own affairs (both internal and external).

Though in 1949, India declared the continuation of her full membership of the Commonwealth of Nations and accepted the British Crown as the head of the Commonwealth, this extra-constitutional declaration does not affect India's membership of the United Nations Organization (UNO) also in no way constitutes a limited on her sovereignty\endnote{4}.

Being a sovereign state, India can either acquire a foreign territory or cede a part of its territory of a foreign state.

\subsection{Socialist}

Even before the term was added by the 42nd Amendment of 1976, the Constitution had a socialist content in the form of certain Directive Principles of State Policy. In other words, what was hitherto implicit in the Constitution has now been made explicit. Moreover, the Congress party itself adopted a resolution\endnote{5} to establish a \quotesingle{socialistic pattern of society} in its Avadi session as early as in 1955 and took measures accordingly. Notably, the Indian brand of socialism is a \quotesingle{democratic socialism} and not a \quotesingle{communistic socialism} (also known as \quotesingle{state socialism}) which involves the nationalisation of all means of production and distribution and the abolition of private property. Democratic socialism, on the other hand, holds faith in a \quotesingle{mixed economy} where both public and private sectors co-exist side by side\endnote{6}. As the Supreme Court says, \quotesingle{Democratic socialism aims to end poverty, ignorance, disease and inequality of opportunity\endnote{7}. Indian socialism is a blend of Marxism and Gandhi-ism, leaning heavily towards Gandhian socialism}. The new economic policy (1991) of liberalization, privatization and globalization has, however, diluted the socialist credentials of the Indian State.


\theendnotes
\cleardoublepage